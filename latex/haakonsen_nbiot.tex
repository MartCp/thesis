\documentclass[USenglish]{ifimaster}  %% ... or USenglish or norsk or nynorsk
\usepackage[utf8]{inputenc}           %% ... or latin1 or applemac
\usepackage[T1]{fontenc,url}
\urlstyle{sf}
\usepackage{babel,textcomp,csquotes,duomasterforside,varioref}
\usepackage[backend=biber,style=numeric-comp]{biblatex}
\usepackage[acronym, xindy]{glossaries}
\usepackage[version=3]{mhchem}

\usepackage{array}
\usepackage{amsmath}
\usepackage[pdftex]{xcolor}
\usepackage[
  pdftitle={INTRODUCTION AND INVESTIGATION},
  pdfauthor={Henning Håkonsen},
  unicode=true,
  colorlinks=true,
  linkcolor=black,
  urlcolor=black,
  citecolor=black,
  breaklinks=true,
]{hyperref}
\urlstyle{same}

\usepackage{emptypage}
\usepackage{fancyhdr}
\usepackage{subcaption}
\pagestyle{fancy}
\fancyhf{}
\fancyhead[LE,RO]{\thepage}
\renewcommand{\sectionmark}[1]{ \markright{\thesection\ #1}{} }
\fancyhead[RE,LO]{\rightmark}

%% Code
\usepackage{listings}
\usepackage{color}
\usepackage{textcomp}
\usepackage[T1]{fontenc}
\usepackage{caption}
\usepackage{float}

\definecolor{jskeywords}{HTML}{E4D00A}% JavaScript keywords
\definecolor{jsextkeywords}{HTML}{FF6700}% JavaScript extended keywords

\definecolor{identifiers}{HTML}{645452} % identfiers
\definecolor{string}{HTML}{B57281} % string literals
\definecolor{allcomment}{HTML}{808080} % comment

\definecolor{nodejs}{HTML}{629755} % Nodejs keywords
\definecolor{testing}{HTML}{4169E1} % Node.js assert, jasmine
\definecolor{express}{HTML}{FF8C69} % Express.js
\definecolor{linenumber}{HTML}{996515} % line number
\definecolor{apricot}{HTML}{98777B} % numbers
\definecolor{linenofill}{HTML}{BEBEBE} % line number fill color
\definecolor{antiquefuchsia}{HTML}{915C83} % braces
\definecolor{ballblue}{HTML}{21ABCD} % braces

\definecolor{captioncolor}{rgb}{0.39, 0.33, 0.32} % caption color
\captionsetup[lstlisting]{font={color=captioncolor, small,tt}}

\captionsetup[lstlisting]{font={color=captioncolor, small, tt}}
\DeclareCaptionFormat{listing}{\vskip1pt#1#2#3}
\captionsetup[lstlisting]{format=listing, singlelinecheck=false, margin=0pt, font={sf},labelsep=space,labelfont=bf}

\lstdefinelanguage{JavaScript}{
  alsoletter={.},
  keywords={arguments,await,break,case,catch,class,const,continue,debugger,default,delete,do,else,enum,eval,export,extends,false,finally,for,function,if,implements,import,in,instanceof,interface,let,new,null,package,private,protected,public,return,static,super,switch,this,throw,true,try,typeof,var,void,while,with,yield}, % JavaScript ES6 keywords
  keywordstyle=\color{jskeywords}\bfseries,
  ndkeywords={add, apply, args, Array, Array.from, Array.isArray, Array.of , Array.prototype, ArrayBuffer, bind, Boolean, call, charAt, charCodeAt, clear, codePointAt, concat, constructor, copyWithin, DataView, Date, Date.now, Date.parse, Date.prototype, Date.UTC, decodeURI, decodeURIComponent, encodeURI, encodeURIComponent, endsWith, entries, Error, Error.prototype, EvalError, every, false, fill, filter, find, findIndex, Float32Array, Float64Array, forEach, FulfillPromise, Function, Function.length, get, getDate, getDay, getFullYear, getHours, getMilliseconds, getMinutes, getMonth, getSeconds, getTime, getTimezoneOffset, getUTCDate, getUTCDay, getUTCFullYear, getUTCHours, getUTCMilliseconds, getUTCMinutes, getUTCMonth, getUTCSeconds, has,hasInstance, hasOwnProperty, ignoreCase, includes, indexOf, indexOf, Infinity, Int8Array, Int16Array, Int32Array, isConcatSpreadable, isFinite, isNaN, IsPromise, isPrototypeOf, Iterable, iterator, join, JSON, JSON.parse, JSON.stringify, keys, lastIndexOf, lastIndexOf, length, localeCompare, map, Map, match, match, Math, Math.abs , Math.acos, Math.acosh, Math.asin, Math.asinh, Math.atan, Math.atan2, Math.atanh, Math.cbrt, Math.ceil, Math.clz32, Math.cos, Math.cosh,  Math.E, Math.exp, Math.expm1, Math.floor, Math.fround, Math.hypot, Math.imul, Math.LN2, Math.LN10, Math.log, Math.log1p, Math.log2, Math.LOG2E, Math.log10, Math.LOG10E, Math.max, Math.min, Math.PI, Math.pow, Math.random, Math.round, Math.sign, Math.sin, Math.sinh, Math.sqrt, Math.SQRT1_2, Math.SQRT2, Math.tan, Math.tanh, Math.trunc, message, multiline, name, NaN, NewPromiseCapability, next, normalize, null, Number, Number.EPSILON, Number.isFinite, Number.isInteger, Number.isNaN, Number.isSafeInteger, Number.MAX_SAFE_INTEGER, Number.MAX_VALUE, Number.MIN_SAFE_INTEGER, Number.MIN_VALUE, Number.NaN, Number.NEGATIVE_INFINITY, Number.parseFloat, Number.parseInt, Number.POSITIVE_INFINITY, Number.prototype, Object, Object, Object.assign, Object.create, Object.defineProperties, Object.defineProperty, Object.freeze, Object.getOwnPropertyDescriptor, Object.getOwnPropertyNames, Object.getOwnPropertySymbols, Object.getPrototypeOf, Object.is, Object.isExtensible, Object.isFrozen, Object.isSealed, Object.keys, Object.preventExtensions, Object.prototype, Object.seal, Object.setPrototypeOf, of, parseFloat, parseInt, pop, Promise, Promise.all , Promise.race, Promise.reject, Promise.resolve, PromiseReactionJob, propertyIsEnumerable, prototype, Proxy, Proxy.revocable , push, RangeError, reduce, reduceRight, ReferenceError, Reflect, Reflect.apply, Reflect.construct , Reflect.defineProperty, Reflect.deleteProperty, Reflect.enumerate, Reflect.get, Reflect.getOwnPropertyDescriptor, Reflect.getPrototypeOf, Reflect.has, Reflect.isExtensible, Reflect.ownKeys, Reflect.preventExtensions, Reflect.set, Reflect.setPrototypeOf, Reflection, RegExp, RegExp, RegExp.prototype, repeat, replace, replace, reverse, search, search, Set, set, setDate, setFullYear, setHours, setMilliseconds, setMinutes, setMonth, setSeconds, setTime, setUTCDate, setUTCFullYear, setUTCHours, setUTCMilliseconds, setUTCMinutes, setUTCMonth, setUTCSeconds, shift, slice, slice, some, sort, species, splice, split, split, startsWith, String, String.fromCharCode, String.fromCodePoint, String.raw, substring, Symbol, Symbol.for, Symbol.hasInstance, Symbol.isConcatSpreadable, Symbol.iterator, Symbol.keyFor, Symbol.match, Symbol.prototype, Symbol.replace, Symbol.replace, Symbol.search, Symbol.species, Symbol.split, Symbol.toPrimitive, Symbol.toStringTag, Symbol.unscopables, SyntaxError, then, toDateString, toExponential, toFixed, toISOString, toJSON, toLocaleDateString, toLocaleLowerCase, toLocaleString, toLocaleString, toLocaleString, toLocaleString, toLocaleTimeString, toLocaleUpperCase, toLowerCase, toPrecision, toPrimitive, toString, toStringTag, toTimeString, toUpperCase, toUTCString, TriggerPromiseReactions, trim, true, TypeError, Uint8Array, Uint8ClampedArray, Uint16Array, Uint32Array, undefined, unscopables, unshift, URIError, valueOf, WeakMap, WeakSet
  }, % JavaScript extended keywords
  ndkeywordstyle=\color{jsextkeywords}\bfseries,
  identifierstyle=\color{identifiers},
  sensitive=true,
  stringstyle=\color{string}\ttfamily,
  morestring=[b]",
  morestring=[d]',
  morestring=[s][\color{string}\ttfamily]{`}{`},
  commentstyle=\color{red}\itshape,
  morecomment=[l][\color{allcomment}]{//},
  morecomment=[s][\color{allcomment}]{/*}{*/},
  morecomment=[s][\color{allcomment}]{/**}{*/},
  emph={app.all, app.delete, app.disable, app.disabled, app.enable, app.enabled, app.engine, app.get, app.listen, app.locals, app.METHOD, app.mountpath, app.param, app.path, app.post, app.put, app.render, app.route, app.set, app.use, express, express.Router, express.static, req.acceptLanguages, req.accepts, req.acceptsCharsets, req.acceptsEncodings, req.app, req.baseUrl, req.body, req.cookies, req.fresh, req.get, req.hostname, req.ip, req.ips, req.is, req.method, req.originalUrl, req.param, req.params, req.path, req.protocol, req.query, req.range, req.route, req.secure, req.signedCookies, req.stale, req.subdomains, req.xhr, res.app, res.append, res.attachment, res.clearCookie, res.cookies, res.download, res.end, res.format, res.get, res.headersSent, res.json, res.jsonp, res.links, res.locals, res.location, res.redirect, res.render, res.sendFile, res.sendStatus, res.set, res.status, res.type, res.vary, router.all, router.METHOD, router.param, router.route, router.use}, % express keywords
  emph={[2]agent.createConnection, agent.destroy, agent.freeSockets, agent.getName, agent.maxFreeSockets, agent.maxSockets, agent.requests, agent.sockets, certificate.exportChallenge, certificate.exportPublicKey, certificate.verifySpkac, child.channel, child.connected, child.disconnect, child.kill, child.pid, child.send, child.stderr, child.stdin, child.stdio, child.stdout, child_process.exec, child_process.execFile, child_process.execFileSync, child_process.execSync, child_process.fork, child_process.spawn, child_process.spawnSync, cipher.final, cipher.getAuthTag, cipher.setAAD, cipher.setAutoPadding, cipher.update, clearImmediate, clearImmediate, clearInterval, clearInterval, clearTimeout, clearTimeout, console, console.assert, console.dir, console.error, console.info, console.log, console.time, console.timeEnd, console.trace, console.warn, decipher.final, decipher.setAAD, decipher.setAuthTag, decipher.setAutoPadding, decipher.update, dgram.createSocket, dgram.createSocket, diffieHellman.computeSecret, diffieHellman.generateKeys, diffieHellman.getGenerator, diffieHellman.getPrime, diffieHellman.getPrivateKey, diffieHellman.getPublicKey, diffieHellman.setPrivateKey, diffieHellman.setPublicKey, diffieHellman.verifyError, dns.getServers, dns.getServers, dns.lookup, dns.lookup, dns.lookupService, dns.resolve, dns.resolve4, dns.resolve6, dns.resolveCname, dns.resolveMx, dns.resolveNaptr, dns.resolveNs, dns.resolvePtr, dns.resolveSoa, dns.resolveSrv, dns.resolveTxt, dns.reverse, dns.setServers, ecdh.computeSecret, ecdh.generateKeys, ecdh.getPrivateKey, ecdh.getPublicKey, ecdh.setPrivateKey, ecdh.setPublicKey, error.address, error.code, error.errno, error.message, error.path, error.port, error.stack, error.syscall, exports, fs.access, fs.accessSync, fs.appendFile, fs.appendFileSync, fs.chmod, fs.chmodSync, fs.chown, fs.chownSync, fs.close, fs.closeSync, fs.constants, fs.createReadStream, fs.createWriteStream, fs.exists, global, http.createServer, http.get, http.globalAgent, http.request, https.createServer, https.get, https.globalAgent, https.request, message.destroy, message.headers, message.httpVersion, message.method, message.rawHeaders, message.rawTrailers, message.setTimeout, message.socket, message.statusCode, message.statusMessage, message.trailers, message.url, module, module.children, module.exports, module.filename, module.id, module.loaded, module.parent, module.require, os.arch, os.constants, os.cpus, os.endianness, os.EOL, os.freemem, os.homedir, os.hostname, os.loadavg, os.networkInterfaces, os.platform, os.release, os.tmpdir, os.totalmem, os.type, os.uptime, os.userInfo, path.basename, path.delimiter, path.dirname, path.extname, path.format, path.isAbsolute, path.join, path.normalize, path.parse, path.posix, path.relative, path.resolve, path.sep, path.win32, process, process.abort, process.arch, process.argv, process.argv0, process.channel, process.chdir, process.config, process.connected, process.cpuUsage, process.cwd, process.disconnect, process.emitWarning, process.env, process.execArgv, process.execPath, process.exit, process.exitCode, process.getegid, process.geteuid, process.getgid, process.getgroups, process.getuid, process.hrtime, process.initgroups, process.kill, process.mainModule, process.memoryUsage, process.nextTick, process.pid, process.platform, process.release, process.send, process.setegid, process.seteuid, process.setgid, process.setgroups, process.setuid, process.stderr, process.stdin, process.stdout, process.title, process.umask, process.uptime, process.version, process.versions, querystring.escape, querystring.parse, querystring.stringify, querystring.unescape, r.clearLine, readable.pause, readable.pipe, readable.push, readable.push, readable.read, readable.read, readable.resume, readable.setEncoding, readable.unpipe, readable.unshift, readable.wrap, readable._read, readStream.bytesRead, readStream.isRaw, readStream.path, readStream.setRawMode, repl.start, request.abort, request.aborted, request.end, request.flushHeaders, request.setNoDelay, request.setSocketKeepAlive, request.setTimeout, request.write, require, require.cache, require.extensions, response.addTrailers, response.end, response.finished, response.getHeader, response.getHeaderNames, response.getHeaders, response.hasHeader, response.headersSent, response.removeHeader, response.sendDate, response.setHeader, response.setTimeout, response.statusCode, response.statusMessage, response.write, response.writeContinue, response.writeHead, rl.clearScreenDown, rl.close, rl.createInterface, rl.cursorTo, rl.emitKeypressEvents, rl.moveCursor, rl.pause, rl.prompt, rl.question, rl.resume, rl.setPrompt, rl.write, script.runInNewContext, script.runInThisContext, server.addContext, server.address, server.address, server.close, server.close, server.connections, server.getTicketKeys, server.listen, server.listen, server.setTicketKeys, server.setTimeout, server.setTimeout, server.timeout, server.timeout, setImmediate, setInterval, setTimeout, socket.addMembership, socket.address, socket.bind, socket.bind, socket.close, socket.dropMembership, socket.ref, socket.send, socket.setBroadcast, socket.setMulticastLoopback, socket.setMulticastTTL, socket.setTTL, socket.unref, stream.Readable, stringDecoder.end, stringDecoder.write, timeout.ref, timeout.unref, tls.connect, tls.createSecureContext, tls.createServer, tls.getCiphers, tlsSocket.address, tlsSocket.authorizationError, tlsSocket.authorized, tlsSocket.encrypted, tlsSocket.getCipher, tlsSocket.getEphemeralKeyInfo, tlsSocket.getPeerCertificate, tlsSocket.getProtocol, tlsSocket.getSession, tlsSocket.getTLSTicket, tlsSocket.localAddress, tlsSocket.localPort, tlsSocket.remoteAddress, tlsSocket.remoteFamily, tlsSocket.remotePort, tlsSocket.renegotiate, tlsSocket.setMaxSendFragment, transform._flush, transform._transform, util.debuglog, util.deprecate, util.format, util.inherits, util.inspect, v8.getHeapStatistics, v8.setFlagsFromString, vm.createContext, vm.isContext, vm.runInContext, vm.runInDebugContext, vm.runInNewContext, vm.runInThisContext, watcher.close, worker.disconnect, worker.exitedAfterDisconnect, worker.id, worker.isConnected, worker.isDead, worker.kill, worker.process, worker.send, worker.suicide, writable.cork, writable.end, writable.setDefaultEncoding, writable.write, writeStream.bytesWritten, writeStream.columns, writeStream.path, writeStream.rows, zlib, zlib.createGunzip, zlib.createGzip, zlib.createInflate, zlib.createInflateRaw, zlib.createUnzip, zlib.deflate, zlib.deflateRaw, zlib.deflateRawSync, zlib.deflateSync, zlib.gunzip, zlib.gunzipSync, zlib.gzip, zlib.gzipSync, zlib.inflate, zlib.inflateRaw, zlib.inflateRawSync, zlib.inflateSync, zlib.unzip, zlib.unzipSync, __dirname, __filename}, % Node.js keywords
  emph={[3] assert, assert.deepEqual, assert.deepStrictEqual, assert.doesNotThrow, assert.equal, assert.fail, assert.ifError, assert.notDeepEqual, assert.notDeepStrictEqual, assert.notEqual, assert.notStrictEqual, assert.ok, assert.strictEqual, assert.throws, describe, toBe, it, xdescribe, beforeEach, afterEach, beforeAll, afterAll, expect, it, xit, xdiscribe, pending, and.callThrough, and.returnValue, and.returnValues, and.callFake, and.throwError, and.stub, .not, .calls.any, .calls.count, .calls.argsFor, .calls.allArgs, .calls.all, .calls.mostRecent, .calls.first, .calls.reset, jasmine.createSpy, jasmine.createSpyObj, jasmine.any, jasmine.anything, jasmine.objectContaining, jasmine.arrayContaining, jasmine.stringMatching, asymmetricMatch,  jasmine.clock, .not.toBeTruthy, .toBeTruthy, .not.toBeFalsy, .toBeFalsy, .not.toBeDefined .toBeDefined, .not.toBeNull .toBeNull, .not.toEqual .toEqual, .not.toBeCloseTo .toBeCloseTo, .not.toContain, .toContain, .not.toMatch, .toMatch, .not.toBeGreaterThan, .toBeGreaterThan, .not.toBeLessThan, .toBeLessThan, .toThrow, .not.toThrow, .toBeNull, .not.toBeNull, .toBeDefined, .not.toBeDefined}, % Node.js Assert, Jasmine, ... keywords
  }

  \lstset{
   basicstyle=\ttfamily,
   language=JavaScript,
   frame=top,frame=bottom,
   breaklines=true,
   showstringspaces=false,
   tabsize=2,
   upquote=true,
   numbers=left,
   numberstyle=\tiny,
   stepnumber=1,
   numbersep=5pt,
   numberblanklines=false,
   xleftmargin=17pt,
   framexleftmargin=17pt,
   framexrightmargin=0pt,
   framexbottommargin=5pt,
   alsoother={.},
   captionpos=t,
   literate=
            *{\{}{{\textcolor{antiquefuchsia}{\{}}}{1}% punctuators
            {\}}{{\textcolor{antiquefuchsia}{\}}}}{1}%
            {(}{{\textcolor{antiquefuchsia}{(}}}1%
            {)}{{\textcolor{antiquefuchsia}{)}}}1%
            {[}{{\textcolor{antiquefuchsia}{[}}}1%
            {]}{{\textcolor{antiquefuchsia}{]}}}1%
            {...}{{\textcolor{ballblue}{...}}}1%
            {;}{{\textcolor{antiquefuchsia}{;}}}1%
            {,}{{\textcolor{antiquefuchsia}{,}}}1%
            {>}{{\textcolor{ballblue}{>}}}1%
            {<}{{\textcolor{ballblue}{<}}}1%
            {<=}{{\textcolor{ballblue}{<=}}}1%
            {>=}{{\textcolor{ballblue}{>=}}}1%
            {==}{{\textcolor{ballblue}{==}}}1%
            {!=}{{\textcolor{ballblue}{!=}}}1%
            {===}{{\textcolor{ballblue}{===}}}1%
            {!==}{{\textcolor{ballblue}{!==}}}1%
            {+}{{\textcolor{ballblue}{+}}}1%
            {-}{{\textcolor{ballblue}{-}}}1%
            {*}{{\textcolor{ballblue}{*}}}1%
            {\%}{{\textcolor{ballblue}{\%}}}1%
            {++}{{\textcolor{ballblue}{++}}}1%
            {--}{{\textcolor{ballblue}{--}}}1%
            {<<}{{\textcolor{ballblue}{<<}}}1%
            {>>}{{\textcolor{ballblue}{>>}}}1%
            {>>>}{{\textcolor{ballblue}{>>>}}}1%
            {=}{{\textcolor{ballblue}{=}}}1%
            {&}{{\textcolor{ballblue}{&}}}1%
            {|}{{\textcolor{ballblue}{|}}}1%
            {^}{{\textcolor{ballblue}{^}}}1%
            {!}{{\textcolor{ballblue}{!}}}1%
            {~}{{\textcolor{ballblue}{~}}}1%
            {&&}{{\textcolor{ballblue}{&&}}}1%
            {||}{{\textcolor{ballblue}{||}}}1%
            {?}{{\textcolor{ballblue}{?}}}1%
            {:}{{\textcolor{ballblue}{:}}}1%
            {=}{{\textcolor{ballblue}{=}}}1%
            {+=}{{\textcolor{ballblue}{+=}}}1%
            {-=}{{\textcolor{ballblue}{-=}}}1%
            {*=}{{\textcolor{ballblue}{*=}}}1%
            {\%=}{{\textcolor{ballblue}{\%=}}}1%
            {<<=}{{\textcolor{ballblue}{<<=}}}1%
            {>>=}{{\textcolor{ballblue}{>>=}}}1%
            {>>>=}{{\textcolor{ballblue}{>>>=}}}1%
            {&=}{{\textcolor{ballblue}{&=}}}1%
            {|=}{{\textcolor{ballblue}{|=}}}1%
            {^=}{{\textcolor{ballblue}{^=}}}1%
            {=>}{{\textcolor{ballblue}{=>}}}1%
            {\\b}{{\textcolor{ballblue}{\\b}}}1% escape sequences
            {\\t}{{\textcolor{apricot}{\\t}}}{1}%
            {\\n}{{\textcolor{apricot}{\\n}}}{1}%
            {\\v}{{\textcolor{apricot}{\\v}}}{1}%
            {\\f}{{\textcolor{apricot}{\\f}}}{1}%
            {\\r}{{\textcolor{apricot}{\\r}}}{1}%
            {\\"}{{\textcolor{apricot}{\\"}}}{1}%
            {\\'}{{\textcolor{apricot}{\\'}}}{1}%
            {\\}{{\textcolor{apricot}{\\}}}{1}%
            {0}{{\textcolor{apricot}{0}}}{1}% numbers
            {1}{{\textcolor{apricot}{1}}}{1}%
            {2}{{\textcolor{apricot}{2}}}{1}%
            {3}{{\textcolor{apricot}{3}}}{1}%
            {4}{{\textcolor{apricot}{4}}}{1}%
            {5}{{\textcolor{apricot}{5}}}{1}%
            {6}{{\textcolor{apricot}{6}}}{1}%
            {7}{{\textcolor{apricot}{7}}}{1}%
            {8}{{\textcolor{apricot}{8}}}{1}%
            {9}{{\textcolor{apricot}{9}}}{1}%
            {.0}{{\textcolor{apricot}{.0}}}{2}%
            {.1}{{\textcolor{apricot}{.1}}}{2}%
            {.2}{{\textcolor{apricot}{.2}}}{2}%
            {.3}{{\textcolor{apricot}{.3}}}{2}%
            {.4}{{\textcolor{apricot}{.4}}}{2}%
            {.5}{{\textcolor{apricot}{.5}}}{2}%
            {.6}{{\textcolor{apricot}{.6}}}{2}%
            {.7}{{\textcolor{apricot}{.7}}}{2}%
            {.8}{{\textcolor{apricot}{.8}}}{2}%
            {.9}{{\textcolor{apricot}{.9}}}{2},%
   emphstyle={\color{express}}, % express
   emphstyle={[2]\color{nodejs}}, % node.js
   emphstyle={[3]\color{testing}}, % jasmine ...
   numberstyle=\normalfont\tiny\textcolor{linenumber} % line number
}

\makeglossaries{}

\title{\acrlong{nb-iot}}
\subtitle{Testing of early deployment}
\author{Henning Håkonsen}

\bibliography{mybib}                  %% Load bib

\loadglsentries[main]{glossary}       %% Load glossary from file
\begin{document}
\frontmatter

\duoforside[dept={Department of Informatics},   %% ... or your department
  program={Network and system administration},  %% ... or your programme
  long]                                        %% ... or long

\tableofcontents

\newpage
\lstlistoflistings

\newpage
\listoffigures

\newpage
\listoftables

\cleardoublepage
%!TEX root = ../main.tex

\vspace*{2cm}
\thispagestyle{plain}

\begin{center}

\phantomsection

\section*{Acknowledgements} \label{section:acknowledgements}
\addcontentsline{toc}{section}{\nameref{section:acknowledgements}}
\end{center}

I would like to thank my supervisors, Ola Martin Lykkja and Yan Zhang, for their support. In addition, I would like to thank Q-Free for giving me the opportunity to work with Ola Martin, as well as supplying me with the necessary equipment to fulfill my work.

A special thank you to Telia which let me perform tests on their network. This gave the thesis an additional dimension and improved the results.

I would also like to thank my family, friends and especially my girlfriend, for supporting me through the last two years completing my master's degree.


\cleardoublepage
%!TEX root = ../main.tex

\vspace*{2cm}
\thispagestyle{plain}

\begin{center}

\phantomsection

\section*{Abstract} \label{section:abstract}
\addcontentsline{toc}{section}{\nameref{section:abstract}}
\end{center}

Today there are approximately 30 billion smart devices in the world. The growth forwards will be exponential and analysis predict that there will be 50 billion smart devices by 2020. There is a demand for a new technology which enables communication with sensors and other low powered devices. This thesis resembles the work and research on a set of technologies suited for this communication, with an emphasis on \acrshort{nb-iot}.

Please visit \url{henninghaakonsen.me} for the latest copy of the thesis, along with the related research and results. The web application is present to support my thesis with extra reading material as well as being a tool for my tests. 


\mainmatter{}

\chapter{Introduction}
\section{Goal}
The goal of this thesis is to introduce you to \acrfull{lpwan} technologies and explore the specifications related to one of them, \acrfull{nb-iot}. There are several claims to \acrshort{nb-iot} - battery lifetime over 10 years, indoor and underground coverage and low cost. Through a cooperation with Q-Free, Telenor and Telia, I was able to do hands-on tests of \acrshort{nb-iot}. The interesting part of this thesis is that I was one of the first to test \acrshort{nb-iot} in Norway and it lead me to a motivational process, but also at times frustrating. There are several papers on how \acrshort{nb-iot} has great power saving features, referring to the \acrfull{3gpp} specification, but there are few discussing how they can be achieved in practice. Several factors will influence the power usage, such as environmental changes, downtime in the network, network configuration and software complexity. I will show you how these factors impact the power consumption and give an overview of the best practice guidelines.

\section{Motivation}
Easily accessed information, connected to all parts of society is a huge motivator for \acrfull{iot} and \acrshort{lpwan}. In the future all things will be connected to the Internet, enhancing applications. The definition of \acrshort{iot} is a group of physical devices, for example, vehicles, monitoring systems, watches and so forth, forming a network. The possibilities this network introduces are incredible, but the path towards this goal has not been easy. There have been attempts of similar networks, but they are usually too fixed or expensive and the competition has been more of an obstacle than an advantage. With several emerging technologies suited for low powered devices, Cisco has made an illustration of the expected growth of devices[\vref{pic:IoT2020}].

\begin{figure}[ht]
  \centering\includegraphics[width=12cm,height=6cm]{/Users/henninghakonsen/Dropbox/Masteroppgave/thesis/latex/images/popTo2300.jpg}
  \caption[\acrshort{iot} Growth]{Cisco's expected growth of devices based on an article anticipating the growth of the population by The United Nations \cite{online:pop2300}\cite{online:IoT2020}}
  \label{pic:IoT2020}
\end{figure}

The payload sent from \acrshort{iot} devices is usually small, typically 100 bytes or less. However, by using \acrfull{lte} for \acrfull{m2m} communication on the current infrastructure leads to high load with all the signaling required. Hence, a simplified technology would allow less signaling and load on the telecoms infrastructure, while also lowering monthly fees as a consequence. Figure \vref{pic:costcomparisonfig}, presents two setups. Up until now, \acrshort{iot} devices have communicated directly or indirectly with Internet like alternative A. The typical scenario has been that the \acrfull{ue} communicate with a more complex device, often a wireless unit of some kind and this device, or an additional device, communicates with the \acrfull{enb} or via a wired connection. While this is an improvement and a step towards a future of connected things - the current mobile technology does not support the expected growth in this area. In alternative b however, the \acrshort{ue}'s can communicate directly with the \acrshort{enb} which reduces complexity, cost and maintenance.

Price is always an important factor when choosing a new product. The typical cost of a mobile subscription with \acrshort{lte} and 5-10GB of data per month cost between 20€ and 50€, supporting 50-100 devices. The cost of a subscription over \acrshort{nb-iot} will according to Telenor and Q-Free be around 0.2€ per \acrshort{ue}. In addition, hardware and installation costs are reduced with \acrshort{nb-iot}. The aim is that an \acrshort{nb-iot} enabled device will cost around 60€. A similar \acrshort{lte} implementation would require a component that does \acrshort{lte} for embedded devices, which costs around 500€. The calculation \vref{equation:ltevslwpa}, of the cost of 300 devices from a parking sensor use case shows that \acrshort{nb-iot} outperforms the current solution. On the left-hand side, you see \acrshort{lte} specific additions related to installation and monthly fees from the \acrshort{lte} embedded device. The monthly fees and maintenance come from the rent of the hardware position. When using the \acrshort{lte} embedded device Q-Free needs to rent a space for it and pay for maintenance and power for that particular place. Note that this is just an outline of one example and certain parameters may change depending on hardware, installation company, and network provider.

\begin{figure}
\centering
\begin{subfigure}{.5\textwidth}
  \centering
  \includegraphics[width=.9\linewidth]{/Users/henninghakonsen/Dropbox/Masteroppgave/thesis/latex/images/communication_pre_nbiot}
  \caption{Setup with private base station}
  \label{fig:sub1}
\end{subfigure}%
\begin{subfigure}{.5\textwidth}
  \centering
  \includegraphics[width=.9\linewidth]{/Users/henninghakonsen/Dropbox/Masteroppgave/thesis/latex/images/communication_nbiot}
  \caption{Setup with \acrshort{nb-iot}}
  \label{fig:sub2}
\end{subfigure}
\caption[Cost comparison figure]{Setups pre and post \acrshort{nb-iot} time\cite{communication_prepost_nbiot}}
\label{pic:costcomparisonfig}
\end{figure}

\begin{table}[H]
\centering
\resizebox{\textwidth}{!}{%
\begin{tabular}{|l|l|l|l|l|l|}
\cline{1-2} \cline{4-6}
 &  &  &  & Setup with private base station & Setup with \acrshort{nb-iot} \\ \cline{1-2} \cline{4-6}
Sensors per base station (LTE) & 100 &  & \textbf{One time cost} &  &  \\ \cline{1-2} \cline{4-6}
Subsciption cost - monthly (LTE) & € 50 &  & Cost of sensor & 50 & 60 \\ \cline{1-2} \cline{4-6}
Installation cost - one time (LTE) & € 1 500 &  & Installation cost per sensor & 5.00 & 0 \\ \cline{1-2} \cline{4-6}
Montly fees and maintenance (LTE) & € 250 &  &  &  &  \\ \cline{1-2} \cline{4-6}
 &  &  &  &  &  \\ \cline{1-2} \cline{4-6}
Number of years & 3 &  & \textbf{Montly cost} &  &  \\ \cline{1-2} \cline{4-6}
Number of sensors & 300 &  & Subscribtion & 0.50 & 0.2 \\ \cline{1-2} \cline{4-6}
 &  &  & Subscribtion over 3 years & 18.00 & 7.2 \\ \cline{1-2} \cline{4-6}
 &  &  & Fees and maintenance & 0.83 & 0.3 \\ \cline{1-2} \cline{4-6}
 &  &  & Fees and maintenance over 3 years & 30.00 & 10.8 \\ \cline{1-2} \cline{4-6}
 &  &  &  &  & \\ \cline{1-2} \cline{4-6}
 &  &  & Total cost per sensor & € 103 & € 78 \\ \cline{1-2} \cline{4-6}
 &  &  & Total cost 300 sensors & € 30 900 & € 23 400 \\ \cline{1-2} \cline{4-6}
\end{tabular}%
}
\caption[Cost comparison]{Cost comparison between \acrshort{nb-iot} and \acrshort{lte} with of 300 devices.}
\label{equation:ltevslwpa}
\end{table}

In addition to lower cost and lower complexity, there is a demand for increased coverage. Vehicles, monitoring sensors positioned several meters underground and devices located in packed cities are dependent of extreme coverage. \acrshort{nb-iot} will add 20 dB to the link margin, which will enable such devices to operate while holding the power usage to a minimum.

Another motivator for \acrshort{iot} is the expected easiness of the setup. With today's technology, you have to use sensors communicating over Bluetooth or Wi-Fi to a common gateway which is connected to the Internet. With a \acrshort{nb-iot} device you will be able to connect directly to Internet, which will encourage many people to engage in this technology, perhaps people without development experience as well. This will be an additional cause of growth in the number of \acrshort{iot} devices and yet another reason why there is a demand to properly implement \acrshort{lpwan}.

The idea of some of the new \acrshort{lpwan} is to use the same hardware as \acrshort{lte} and if the customer wants it they will also support software to enable an \acrshort{iot} platform. The most promising technology for \acrshort{m2m} communication is \acrshort{nb-iot} and is enabled with a \acrshort{lte} core network. I will discuss how to deploy \acrshort{nb-iot} in section \vref{section:nb-iot-deployment}. This new technology supports an extreme amount of devices as well as them being in a secure environment. Security is a big concern when discussing general network related topics and I will introduce you to some security measurements provided by \acrshort{nb-iot}.

\section{Current status}
\subsection{Internet}
The main idea of the Internet today is quite similar to the past, but the scale and reach has outgrown what anyone thought could be achieved. In 2015 Google's data centers achieved high speed transfers up to 1 Petabit. "According to Vahdat, that is enough bandwidth for more than 100.000 servers to exchange data at 10 Gbps each, or transmit all scanned contents of the Library of Congress in under one-tenth of a second"\cite{online:petabitGoogle}. The reason for these data centers is the use of online resources. Offices, as well as home users require more bandwidth and reliability of the services provided from e.g. Google, entertainment, data storage and so forth.

\subsection{Mobile networks}
Norway's GSM network came online in 1993 \cite{online:gsmTelenorNetcom} and Norway has pushed the technology to reach higher standards from this time. The evolution of wireless radio technology has usually been focused on making it faster, while not prioritizing battery lifetime and simplicity. However, because this technology is used in mobile devices, it is for most users good news. \acrshort{lte} offers high \acrfull{ul} and \acrfull{dl} speeds and good coverage. One problem with \acrshort{lte} and older wireless technologies is that it consumes a lot of power. People are experiencing high speed transfers to their mobile devices, but at a cost draining their batteries. In the future, power efficiency has to be one of the main features of \acrshort{lpwan} established. This is especially the case for sensor networks as these devices need to be connected to a mobile network. Such sensors are often established in remote locations without steady power and using \acrshort{lte} would give the devices short lifetime.

\subsection{A closer look at \acrshort{lte}} \label{ssection:lte}
A \acrshort{lte} network consist of some particular nodes for the network to connect to \acrshort{ue}'s and Internet. This architecture is called \acrfull{epc} \cite{online:epc} and this section will give you a get a better look at some of the attributes of \acrshort{lte} and the parts which make the \acrshort{epc} network. \acrshort{lte} was introduced in release 8 of the \acrshort{3gpp} in 2008. In a summation by Ericsson, they state some facts about \acrshort{lte} from the initial release. \acrshort{lte} from release 8 supported 100 Mbps \acrshort{dl} and 50 Mbps \acrshort{ul} (Not peak speeds), reduced latency (down to 10ms) and was cost-effective to set up\cite{online:lteIntroduction}. As of release 10 by \acrshort{3gpp} (year 2011) the speeds were at astonishing 1 Gbps \acrshort{dl} and 500 Mbps \acrshort{ul}, however because of overhead in the network it is not suited for low powered devices running on batteries.

\subsection{Introduction to key features of \acrlong{epc}}
\acrshort{lte} is composed by 5 components which interact with each other. \acrshort{enb}, \acrshort{mme}, \acrshort{hss}, \acrshort{s-gw} and \acrshort{p-gw} constitutes \acrshort{epc}. See outline of \acrshort{lte} topology in figure \vref{online:lteTopology}.

\begin{figure}[H]
  \centering\includegraphics[width=12cm,height=6cm]{/Users/henninghakonsen/Dropbox/Masteroppgave/thesis/latex/images/epc.jpg}
  \caption[\acrlong{epc} overview]{\acrlong{epc} overview \cite{online:lteTopology}}
  \label{online:lteTopology}
\end{figure}
%%http://www.3gpp.org/technologies/keywords-acronyms/100-the-evolved-packet-core
%%http://www.radio-electronics.com/info/cellulartelecomms/lte-long-term-evolution/sae-system-architecture-evolution-network.php

\subsubsection{\acrshort{enb}}
The \acrfull{enb} is what most people refer to as the base station of a mobile network. It is responsible for distributing signal to the users and deal with \acrshort{ul} and \acrshort{dl} data. The \acrshort{enb} can be set up to suit different scenarios by adding frequency bands. For \acrshort{lte}, most network providers use 800, 900, 1800 and 2100 Mhz, while for \acrshort{nb-iot} most companies will use 800 Mhz because at this frequency the signal will reach longer.

\subsubsection{\acrshort{mme}}
\acrfull{mme} is the main provider of signaling in \acrshort{lte}. \acrshort{mme} is connected to \acrshort{enb}, \acrshort{hss} and \acrshort{s-gw}, through S1-MME, S6 and S11 interfaces. It is in charge of authentication in cooperation with \acrshort{hss} and terminal-to-network negotiation.

\subsubsection{\acrshort{hss}}
\acrfull{hss} is the main database for information in the network. It is a joint service originating from \acrfull{hlr} and \acrfull{auc}. \acrshort{hss} keeps information about subscriber information, that being user identification, addressing and profiles for a subscriber. Profiles describe how the network should perform for this special user and may include parameters like \acrfull{qos} and special modes, or states for a subscriber. \acrshort{hss} also holds information about authentication between the network and \acrshort{ue}s.

\subsubsection{\acrshort{s-gw} and \acrshort{p-gw}}
\acrfull{s-gw} and \acrfull{p-gw} deal with the user data plane and transports IP packets from \acrshort{epc} to external networks. The \acrshort{s-gw} maintains paths from \acrshort{enb}s to \acrshort{p-gw}s.

\section{Future mobile networks} \label{section:futureapplications}
For the past ten years there has been a rise in sensor activity. In the evolution of \acrshort{iot} devices, some has ported to \acrshort{lpwan}. However, as stated in the motivation, \acrshort{iot} devices rely on good coverage and stability. Many people think of home electronics when discussing \acrshort{iot}, but this is not the main goal of \acrshort{lpwan} devices. Oslo's public transport operator (Ruter) has payment cards communicating with ticket readers over RFID, which in turn communicate with Internet. The Norwegian transport agency uses \acrfull{dsrc} in their tolling stations to communicate with cars, which is a standard developed by CEN TC278 WG1\cite{EFCCENTC29:online}. The following sections gives examples of applications related to future mobile networks.

\subsection{Applications} \label{ssection:applications}
Already back in 2009, a forum called \acrfull{oecd} discussed ways to take advantage of sensors to support a green growth \cite{online:industryApplications}. This forum has members from all over the world, including Norway. The idea is that sensors can surveil and monitor emission numbers as well as act upon them. \acrshort{oecd} came up with a group of applications which were important to investigate. Some of them are smart grids and energy control systems, smart buildings, transport and logistics, agriculture and other general industrial applications. Together these fields combine most of the energy use today and with smarter systems, power consumption can be reduced.

With this in mind, it is clear that many of the fields \acrshort{oecd} discussed has been implemented, or is in the deployment stage. However, the current status of industry activities related to these fields are running over 2G/3G/\acrshort{lte} networks. These networks provide average coverage, cost and power consumption, hence with billions of devices enrolling, this is not a sustainable model.

\acrfull{gsma} has made an overview of the most applicable areas of use. Figure \vref{pic:lpwan}, presents some bullet points for each area, and the next section will introduce you to some of the areas.

\begin{figure}[ht]
  \centering\includegraphics[width=\textwidth,height=8cm]{/Users/henninghakonsen/Dropbox/Masteroppgave/thesis/latex/images/lpwaArea.jpg}
  \caption[\acrshort{lpwan} applications]{\acrshort{lpwan} applications \cite{online:lpwaFuture}}
  \label{pic:lpwan}
\end{figure}

\subsubsection{Utilities and smart cities}
This domain resolves to what is investigated in this thesis. Cities are getting smarter by the day and \acrshort{lpwan} can be the answer to a large sensor network for monitoring and managing cities. \acrfull{its} is an initiative to control and manage traffic. The system provides communication between cars, trucks and sensors. The sensors are mounted in traffic lights, signs and other objects known in the transport field. The connected devices can adopt to the environment and help society in several ways. Emergency units can set a route to an accident and the overlaying \acrshort{its} system will clear the way remotely to enable the emergency unit to quickly arrive at the location.

Examples of other usage areas of sensors in cities are waste areas, power monitoring and \ce{CO2} monitoring. Operations can heavily benefit from more information, as well as automated procedures. There are many use cases where monitoring can be difficult and costly. With simple and cheap devices one could monitor hundreds of different things. As a result the work effort could decrease and the processes can be more efficient. A good example is garbage disposal. If garbage containers were fitted with measuring sensors, the control system can be notified when it is time to empty the garbage. The use case for this will probably be cities, bigger companies and industrial sites, however it might be used for private residents as well.

These devices need to be cost efficient since they will be deployed in large scale. In 2014 a power supplier in Norway called Lyse began installing smart power meters in over 140 000 households \cite{online:lyseAMS}. These devices communicate with the mobile network in the area over a proprietary standard. While it is important to explore new solutions, this is one of many examples where standardized solutions will greatly outperform proprietary solutions.

\subsubsection{Parking - a realistic use case} \label{ssection:sensoroutline}
Big cities are investing in smart parking even though cars may be banned from the city centres in some countries. People will still prefer traveling with car and facilitating for efficient parking at a nearby location in these cases is crucial. These parking lots can be managed with sensors communicating over \acrshort{lpwan}. For Q-Free's next generation parking sensor, they are mainly focusing on \acrshort{nb-iot}. The sensor is housed in by a small plastic case which will be mounted in flush with the ground. It withstands extreme forces and detects cars by magnetometer and pulsed Doppler radar. The housing is the most expensive part of the sensor, being able to withstand rough conditions for over 10 years. The idea is that the sensor is drilled into the ground, and will communicate over \acrshort{nb-iot}. The \acrshort{ue} will send status of the occupancy of the parking spot with a fixed interval, as well as reporting parking events when they occur. See \ref{pic:parkingsensor} for a photo of the parking sensor from Q-Free.

As of 2016 the parking sensors communicated to a common gateway (\acrshort{lte} embedded device), a device which communicates with the sensors using a proprietary narrowband communication technology in the ISM band, and to the Internet over \acrshort{lte}. This works fine, but as you might have realized, the cost of the devices themselves and managing them are lowered significantly with \acrshort{lpwan} like \acrshort{nb-iot}. Q-Free has chosen to go with \acrshort{nb-iot} since this is the most promising technology providing the necessary requirements for their application. The communication between the sensor and the server is outlined in figure \vref{pic:sensorcommunication}. The sensor sends data to the eNB which in turn sends the data to the \acrshort{iot} platform. If a server has been authenticated within the platform the message is sent to the server. The server receives the packet and can display the message however necessary.

\begin{figure}[H]
  \centering\includegraphics[width=12cm,height=8cm]{/Users/henninghakonsen/Dropbox/Masteroppgave/thesis/latex/images/parkingsensor.jpg}
  \caption[Parking sensor]{Parking sensor from Q-Free\cite{person:ola}}
  \label{pic:parkingsensor}
\end{figure}

\begin{figure}[H]
  \centering\includegraphics[width=12cm,height=8cm]{/Users/henninghakonsen/Dropbox/Masteroppgave/thesis/latex/images/sensorcommunication.jpg}
  \caption[Outline of sensor communication from Q-Free]{Outline of sensor communication \cite{person:ola}}
  \label{pic:sensorcommunication}
\end{figure}

\subsubsection{Logistics}
Today parcel tracking is a huge success and some companies are using sensors in their trucks for real time tracking. Using \acrshort{lpwan} networks will reduce cost of the devices and with long battery lifetime the sensors can be reused. In addition, the coverage for tracing the parcels will have to be great. If you want real time, or near real time tracking, the sensor would have to send position data often and this will drain the battery substantially. There are some pros to use \acrshort{lpwan} networks for this kind of information collection, but this is probably not the biggest \acrshort{iot} area.

\subsubsection{Industrial}
Industry is a wide area and covers sites like gas stations, construction sites and factories. \acrshort{gsma} also includes vending machines in this category \cite{online:lpwaFuture}. With sensors communicating over \acrshort{lpwan} industry stations can be more automated and it is easier to surveil the systems. Downtime on such systems means lost revenue and efficiency. Low cost sensors which are wireless will hopefully have good coverage and uptime so that these systems run better. This is probably one of the areas where these sensors will be used more frequently since these systems needs continuous monitoring. \acrshort{lpwan} will also solve coverage issues related to industry use. As mentioned good coverage would prevent downtime, but it is also important to notice where some of these sensors will be placed. Factories and underground are some of the locations where industry is often located and therefor coverage is especially important for industry usage.

\subsubsection{Consumer}
Many people think of consumer products when talking about \acrshort{iot}. There is growth in smart watches and smart wearables, and the biggest problem with these devices today is the battery life. This is mainly due to the extreme cost of communicating either with your mobile telephone over bluetooth or over \acrshort{lte}. It will be interesting when these watches will use \acrshort{lpwan}. Some wearables, such as your watch may use a standard with higher bandwidth than e.g. \acrshort{nb-iot}. For normal use, one would probably prefer \acrshort{catm} which offers bandwidth up to 1\acrshort{mbit} and could possibly provide music playback and phone calls.

Another interesting use case is smart homes with better alarms, water and gas metering, light control and so forth. While this usually has communicated over Wi-Fi and proprietary \acrfull{ism} solutions, there is a potential market where tradition Wi-Fi is not possible or preferable, hence \acrshort{lpwan} can be used.

\subsubsection{Medical}
A potential step into a healthier society is for medical clinics to be able to monitor and give realtime consulting to their patients. The patients device could automatically uploaded real time data to their doctor, enabling them to uncover symptoms or deceases at an early stage and by this increase the average lifetime. One can also collect a lot of data for patients with a special decease and research for a cure. For this to be realistic one really needs \acrshort{lpwan} since these sensors will have to be cheap and the stability has to be good.

We are going into an era where the number of elderly will increase heavily worldwide and should seize the opportunity to use this technology to take care of the elderly. A person which needs attention, but can live alone, would prefer an electronical pill dispenser and a system to communicate with the care center. The person could also raise an alarm with this system to alert the care takers. In this way the person will probably have a better life and the cost will stay lower.

\section{Developing wireless technologies} \label{section:wirelesstech}
\subsection{The general idea}
As mentioned in the motivation the general idea of new wireless technologies is to reuse the current \acrshort{lte} infrastructure. One may also use \acrshort{gsm}, as the current advantage of this network is that coverage in some areas are better at the moment. However, \acrshort{gsm} and other 2G technologies are being phased out and in a couple of years \acrshort{lte} will be deployed at all locations which GSM covers, meaning better speeds, \acrshort{ue} density and coverage. Coverage and power consumption are key features of developing wireless technologies and in combination with ease of use it is the reason why some technologies stick while others fade away. Ease of use is a requirement for \acrshort{m2m} communication since the amount of data per packet is relatively small. For some of the technologies, guarantied delivery is not important, but it is important to acknowledge that fast delivery will be a requirement for some applications which uses real time monitoring.
\acrshort{nb-iot} will not suffice for these applications and it is necessary to consider using technologies in parallel or speed up the transmission of e.g. \acrshort{nb-iot}. In the future this may not be a problem considering the pace of wireless evolution, but nevertheless it is important to not suppress the issue.

The following subsections introduces the current contenders in the \acrshort{lpwan} market.

\subsection{eMTC/LTE-M1 and \acrshort{nb-iot}}
\acrfull{emtc} was standardized in the 12th release of \acrshort{3gpp} and updated with \acrshort{nb-iot} in the 13th release. It is meant as an high bandwidth alternative to \acrshort{nb-iot} and is often referred to as \acrshort{lte-m1} or \acrshort{catm}.

Both standards are implemented using the current \acrshort{lte} infrastructure, but they differ in signal power, bandwidth and the number of bands used in an \acrshort{enb}. Figure \vref{figure:legacyWire} shows that \acrshort{lte-m1} provides bandwidth up to 1\acrshort{mbit} while \acrshort{nb-iot} peakes at around 200\acrshort{kbit} (rates from the release they were standardized). The coupling loss of \acrshort{lte-m1} is said to be 12\acrshort{dbm} better than \acrshort{lte}, and \acrshort{nb-iot} 20\acrshort{dbm}. The details about how \acrshort{nb-iot} uses the current infrastructure will be coverd in section \vref{section:nb-iot}.

\subsection{LoRa}
\acrshort{lora} \cite{online:LoRa} is the most popular proprietary solution for enabling \acrshort{iot} devices. It is currently deployed in many cities and uses a proprietary solution. The devices which has a \acrshort{lora} chip uses RF or WiFi to communicate with a common gateway, typically a "cell" tower. The current range is up to 15km, but in dense areas the range only covers 2-5km. The bandwidth is low, but in light of the messages being passed to the gateway that is fine. Figure \vref{figure:LoRa} presents the overlaying infrastructure of a \acrshort{lora} network. It uses many of the same features as mentioned in the motivation where the devices communicate with a middle box (concentrator/gateway), which in turn communicates with Internet over 3G or ethernet.

\begin{figure}[ht]
  \centering\includegraphics[width=\textwidth,height=8cm]{/Users/henninghakonsen/Dropbox/Masteroppgave/thesis/latex/images/lora.jpg}
  \caption[\acrshort{lora} overview]{\acrshort{lora} overview \cite{online:LoRa}}
  \label{figure:LoRa}
\end{figure}

The \acrshort{lora} solution provides \acrshort{lpwan} at relatively low cost and gives average coverage compared to \acrshort{nb-iot} and \acrshort{lte-m1}. However, \acrshort{lte} is being deployed globally and if the coverage is good enough it will be more costly to provide \acrshort{lora} networks as well as \acrshort{nb-iot} or \acrshort{lte-m1} networks.

\begin{figure}[ht!]
  \centering\includegraphics[width=\textwidth,height=8cm]{/Users/henninghakonsen/Dropbox/Masteroppgave/thesis/latex/images/lpwaKeyFeatures.jpg}
  \caption[\acrshort{lpwan} technologies]{\acrshort{lpwan} technologies \cite{online:legacyWire}}
  \label{figure:legacyWire}
\end{figure}

\chapter{A dive into NarrowBand IoT} \label{section:nb-iot}

I will focus on a general approach towards \acrshort{nb-iot} as technology, and test both Telenor and Telia's solution. Section \vref{section:testing}, discuss how and where the tests were conducted as well as examples and results.

Telenor started their \acrshort{nb-iot} network test late 2016, with hardware mainly from Huawei. The goal of the test was to set up a \acrshort{nb-iot} network and test the communication with Q-Free's parking sensors and run own tests before releasing the network to the public. At the same time, Telia had just announced that they had deployed their \acrshort{nb-iot} network, but it was not yet production ready. At the time of delivering this thesis, neither Telenor or Telia had \acrshort{nb-iot} in production. Without any reference to a specific date, they only said that the network would be deployed within a short period of time which indicates that the networks were not in production at the time of the tests, but at a stable state.

\acrshort{nb-iot} is the most promising \acrshort{lpwan} solution and the focusing technology in this thesis. The standard is made by \acrshort{3gpp} and \acrfull{etsi}, and was launched late summer of 2016 with the 13th release of The Mobile Broadband Standard. The technology takes advantage of the current \acrshort{lte} infrastructure in a very efficient way. The deployment requires no extra hardware, so the software necessary can be implemented into current hardware. The following sections introduces the structure of \acrshort{nb-iot} and the design requirements. I have not emphasized security in this thesis and it will only be mentioned in the appropriate sections.

\section{System design}
In a system design process it is important to analyze the use of the new technology. It is crucial to understand how people and the industry will use \acrshort{nb-iot}. A good example of a standard which is causing problems today is \acrshort{ipv4}. When this standard was introduced, no one could predict the growth of online devices and for this reason \acrshort{nb-iot} has some specific system design features which hopefully will cover its use for foreseeable years ahead.

\begin{itemize}
  \item Link budget improvement \newline
  The sensors and devices in mind when deploying \acrshort{nb-iot} need better signal power than the average phone or \acrshort{lte} modem. As mentioned these sensors will be used underground, indoors and areas normally without cell reception. It is therefor important that the signal is suited for this use. The goal is to reach 20dB extended signal power compared to \acrshort{lte}. Not only does this extend the range that these devices can be deployed, but they can emit transmissions with lower power.

  \item \acrshort{ue} density \newline
  Cisco predicts that each each household has on average 4 devices\cite{online:IoT2020}. According to Oslo council there are approximately 332 568 households per 1.1.2016 \cite{online:husholdningstatistikk}, meaning that the device pool would be around 1.3 million devices only for house holds. Taking into account that these devices will be broadly used by the industry as well, the device density will be very high. To put this into perspective we can investigate the device density of one \acrshort{gsm} channel deployed in the 900 mHz frequency band. For each \acrshort{gsm} cannel (200 KHz) there are 8 time slots, giving 8 concurrently connected devices. In one cell tower there are approximately 8 channels, and they are often split between mobile providers. Assuming that all of them belong to one provider, one cell tower can support up to 64 devices concurrently. One \acrshort{gsm} cell tower will provide coverage for the surrounding 1km, and since cell towers to close to each other will cause interference only one cell tower can provide coverage for 1km in radius. \acrshort{lte} has much more resources and can provide coverage for more devices, using time and wave division multiplexing.

  There is however a presumption to why \acrshort{nb-iot} can support so many devices per carrier. In the specifications one \acrshort{nb-iot} channel can support more than 50 thousand devices and keep in mind that \acrshort{nb-iot} can be deployed in one \acrshort{gsm} channel, which only supported 8 devices. The way this is supposed to work is due to the infrequent updates from the devices. One device might send a message every second hour and another might even send messages only once every day. One scenario where this might cause problems is if every programmer/company sends updates on specific times. Typically one would want updates at the hour marker(e.g. 14:00, or 15:00). If thousands of devices do this at the same time, congestion and latency will definitively affect the performance for the devices.

  \item Low complexity \newline %%low cost
  A key feature for the devices supporting \acrshort{nb-iot} is low complexity level. The new standard is less intricate than ordinary \acrshort{lte} which means that the devices can be less complex, resulting in cheaper hardware. This is an important factor for success since these devices need to be cheap to be broadly deployed. The complexity in this technology resides in the core mobile network for it to support many devices. Other than this, the technology supports average speed and latency, perfect for monitoring purposes.

  \item Low power \newline
  The power consumption of the communication needed for \acrshort{lpwan} need to be low for the estimated battery lifetime of a device to be kept. The goal is over 10 years of lifetime, and since these devices probably will be very cheap, it might be cheaper to change the device instead of changing the battery.

  Low complexity and better coverage will help the device to consume less power, but it will not enable the device to operate for 10 years. The system design includes a special operation mode, called \acrlong{psm}. I will elaborate how this key feature works in section \vref{ssection:psm}.

  \item Latency \newline
  For most applications applied to this new technology, latency is not a key feature. One would like the device to send frequent, or infrequent messages to a server or another device, but the time used is not important. However, what is actually a long time? A ping request on a normal wired connection today uses around 5-100ms, and even less for fiber connections. On mobile networks like \acrshort{lte}, the usual ping time is on average higher than on a wired connection, but usually it does not exceed 100ms. \acrshort{nb-iot} aims at a peak latency of 10 seconds. In comparison that is 100 times longer than a normal \acrshort{wan} connection, but it should suffice for most applications using \acrshort{lpwan}. The latency could probably be lowered, but it might interfere with other design goals, such as device density.

  \item Security \newline
  A big issue in Internet today is security, specially \acrshort{ddos} attacks. These attacks are more frequent and is an attack form which enables thousands of machines to continually spam one or several IP addresses, which in practice disables the application. In the fall of 2016, the DNS service provider "Dyn" was down because of a \acrshort{ddos} attack. Being a big DNS server, Dyn's downtime impacted major Internet platforms in the U.S. and Europe\cite{online:ddosAttack}. Since an attack like this needs a huge device pool it is likely to think that \acrshort{iot} devices are a potential tool for hackers. If hackers could access millions of devices with Internet access, they could easily perform global \acrshort{ddos} attacks which would halt our infrastructure, in turn disabling many of us to do our jobs.

  With this in mind, the engineers and designers of \acrshort{nb-iot} needed to consider security issues like this. The concept '\acrshort{iot} platform' is introduced in \acrshort{nb-iot}, and this platform can be used to contribute to security measures in cooperation with \acrshort{mme} and \acrshort{hss}. The system requirement of \acrshort{nb-iot} is that no one should be able to directly access a \acrshort{nb-iot} device. One additional security feature in general with mobile networks is SIM cards. Like cell phones, \acrshort{nb-iot} devices will be fitted with a SIM card, which will enable the core network to authenticate the device, and visa versa. The SIM card has a subscription connected to the service provider, hence making a strong authentication scheme.
\end{itemize}

The current design is a general approach to the requirements of low-powered devices. Section \vref{ssection:applications}, presented several applications related to \acrshort{lpwan} and \cite{ASurveyo24:online} has investigated traffic related issues to \acrshort{m2m} communications over \acrshort{lte}. The following list is a summary of the most essential traffic patterns in \acrshort{m2m} communications\cite{ASurveyo24:online}.

\begin{itemize}
  \item \acrshort{ul} traffic dominated, because most applications today are focusing on data gathering rather than data control.

  \item Traffic generated more uniformly through the day by \acrshort{m2m} communication, compared to normal mobile communication.

  \item Traffic can be periodic, which is related to meatering applications which generate data at specific intervals.

  \item Some applications react to events and generates larger volums of data, hence traffic can be bursty.

  \item The mobility of \acrshort{m2m} devices is lower than conventional mobile devices. \acrshort{ue}'s will for the most of the time remain at the installation site. However, for some applications like wearables this is not true.

  \item For many applications \acrshort{qos} will differ. Alarm systems requires higher uptime, better security and increased \acrshort{tsr}, while wearables requires excellent handovers between different \acrshort{enb}'s and reliable \acrshort{dl} data control.
\end{itemize}

\section{Deployment} \label{section:nb-iot-deployment}
There are numerous articles describing the physical layer of \acrshort{nb-iot}, so it is not emphasized in this thesis. For further details about the physical layer, I recommend reading "A Primer on 3GPP Narrowband Internet of Things"\cite{APrimero97:online}.

There are currently three ways to deploy \acrshort{nb-iot} and the available deployment modes are standalone, in-band and guard-band - see figure \vref{figure:nbiot-operationmodes} for illustrative definition. Standalone operation uses a dedicated GSM/\acrshort{lte} channel, which allows NB IoT to utilize previously used GSM frequencies with their channel bandwidth of 200 kHz, thus allowing for 10 kHz guard bands below and above the \acrshort{nb-iot} channel.

In-band operation utilizes bandwidth within one \acrshort{lte} carrier, and allocates one \acrfull{prb} of a \acrshort{lte} channel. This is the most technical and advanced solution and requires more work by the network provider. The reason for this is because of interference between the \acrshort{lte} and \acrshort{nb-iot} sections in the carrier.

Guard-band operation makes use of the bands not in use by \acrshort{lte} due to interference between \acrshort{lte} carriers. This in-between section is called a guard-band and guards the carries for interfering with each other. This band can be used by \acrshort{nb-iot} and is a solution where one would want to utilize all resources of a cell tower.

A \acrshort{nb-iot} \acrshort{ue} is not required to be configured specifically for either deployment schemes. The \acrshort{ue} searches for a carrier on a 100kHz raster, which implies that when deploying in-band mode the anchor carrier can only be placed in certain \acrshort{prb}'s\cite{APrimero97:online}.

\begin{figure}[ht]
  \centering\includegraphics[width=12cm,height=4cm]{/Users/henninghakonsen/Dropbox/Masteroppgave/thesis/latex/images/nbiot-operationmodes.jpg}
  \caption[\acrshort{nb-iot} deployment overview]{\acrshort{nb-iot} deployment overview \cite{online:legacyWire}}
  \label{figure:nbiot-operationmodes}
\end{figure}

The choice between in-band, guard-band and standalone operation is based on the network providers frequency budget in the given area and prioritisation of traffic. The following paragraphs introduces which operation mode Telenor and Telia has used.

\paragraph{Telenor} has deployed their \acrshort{nb-iot} network in stand-alone operation mode, which means that they are using one dedicated \acrshort{lte} channel\cite{person:ola}, hence the network does not suffer from noise from \acrshort{lte} and vice versa. The results showed that the stability of the network was good and performed close to what the specification dictates.

\paragraph{Telia} has deployed their \acrshort{nb-iot} network in in-band operation mode which means they occupy one \acrshort{prb} in an existing \acrshort{lte} channel. As discussed earlier this results in potential frequent interference between \acrshort{nb-iot} and \acrshort{lte} devices. Using Telia's network revealed a slightly more unstable behavior which might be related to the fact that they are using in-band operation mode. On the other side, this setup is closer to what one expect the situation to be in a production network with other users, hence the performance difference is arguably logical.

\section{\acrshort{iot} platform} \label{section:iotplatform}
\acrshort{iot} platform can be rewarding for both customer and network providers. It may enhance the security of the communication. However, in order to utilize the full potential of an \acrshort{iot} platform the network provider will have to open the content of the data and do a repacking of the data. In my opinion, this is not as secure as an end to end transmission. Another motivation for a network provider to promote the \acrshort{iot} platform is that if they have access to the customer's data they can perform analysis on it and give customer based subscriptions or potentially sell your information. A good example of usage of these kinds of data is the possibility to stream music for free, which was enabled by Telenor and Telia in 2018. The networks has to view the data to know that the data is related to music, i.e. Spotify, Tidal, to give the customer this data for free. Likewise the network providers will have access to the data in the \acrshort{iot} platform which is not preferable for many customers. However, if your application sends non-confidential information and you want the application up and running quickly the \acrshort{iot} platform might enhance your application. There are always tradeoffs to using an abstract implementation, like high order programming languages or something like an \acrshort{iot} platform.

\section{Introduction to energy saving} \label{section:energysaving}
Power saving is an important feature of \acrshort{m2m} communication. Several techniques are used to reduce power consumption, and \acrfull{drx} and \acrfull{psm} are two of them. \acrshort{drx} has existed for 20 years and is implemented in \acrshort{lte}, while \acrshort{psm} was introduced with the development of \acrshort{lpwan}. These two modes, together with paging introduces an improved network, especially suited for sensors. In figure \vref{figure:nb-iot-timeconverge} you can see an outline of the communication process for a device. \acrfull{rrc} connected mode is the mode where the device actually can communicate with the network. The following procedures shows a possible time convergence graph for a device. The next sections explain the techniques used, as well as the specific time periods.

\begin{figure}[ht]
  \centering\includegraphics[width=0.9\textwidth,height=6cm]{/Users/henninghakonsen/Dropbox/Masteroppgave/thesis/latex/images/nb-iot-timeconverge.jpg}
  \caption[\acrshort{nb-iot} transmit overview]{\acrshort{nb-iot} transmit overview \cite{person:ola}}
  \label{figure:nb-iot-timeconverge}
\end{figure}

\subsection{Prerequisites} \label{ssection:prerequisites}
To get a good understanding of the results, it is important to look closely at the outcome of the results and understand the meaning. The following sections describe details about how \acrshort{nb-iot} works and what the specifications are saying. As stated, I will mostly focus the tests on signal power and power consumption, which is closely related. Signal power is measured in \acrfull{dbm} and provides a relation between the distance between the \acrshort{enb} and the \acrshort{ue}, and the transmit power of the \acrshort{ue}. In theory, a device will transmit with the relative strength according to the signal power, so that the signal is received at the \acrshort{enb} with the same level.

I will use \acrshort{mw} throughout this thesis and the equation, \vref{equation:prerequisites}, shows the relation between \acrfull{dbm}, \acrfull{ma} and \acrfull{mw}.

\begin{equation} \label{equation:prerequisites}
\begin{aligned}
  X\quad \acrshort{dbm} => 10^{X\quad \acrshort{dbm} / 10}\quad \acrshort{mw} \\
  I\quad \acrshort{ma} => (I\quad \acrshort{ma} * V)\quad \acrshort{mw} \\
\end{aligned}
\end{equation}

\acrshort{nb-iot} operates between ${-40 \acrshort{dbm}}$ and ${+23 \acrshort{dbm}}$, where theoretically ${-40 \acrshort{dbm}}$ equals ${10^{-4} = 0.0001 \acrshort{mw}}$ and ${+23 \acrshort{dbm}}$ equals ${10^{2.3} = 200 \acrshort{mw}}$.
The chip's manual states that ${-40 \acrshort{dbm}}$ equals ${74 \acrshort{ma} * 3.3V = 244 \acrshort{mw}}$, while ${+23 \acrshort{dbm}}$ equals ${220 \acrshort{ma} * 3.3V = 726 \acrshort{mw}}$.

\subsection{\acrshort{rrc} connected mode} \label{ssection:rrc}
This mode is used when the device wants to communicate with the network, either \acrshort{ul} or \acrshort{dl}. The time spent in \acrshort{rrc} connected mode can vary, but the default \acrshort{nb-iot} timer is 20 seconds. In this mode the device will use a lot more power, hence moving away from this mode as soon as the transmit or receive process is finished is important. When the chip is in active/connected mode it uses ${6 \acrshort{ma} * 3.3V = 0.0198 \acrshort{mw}}$ \cite{datasheet:ubloxchip}.

When the transmit operation is finished the chip will stay in connected mode until the time period ends. There is an optional flag, \acrfull{rai}, in the uplink datatransfer frame which puts the device into \acrshort{psm} directly after the transmit operation. The network needs to support this action to give the \acrshort{ue} permission to sleep. It is the network that decides what the \acrshort{ue} should do according to the network and the device's status. Given that the network supports \acrshort{rai} and this is used the battery life will be extended, since the \acrshort{ue} only sends data for a short period rather than waiting for downlink data which in many applications is not needed.

If \acrshort{rai} is not set the device will stay in \acrshort{rrc} connected mode until it is notified otherwise from the network. This process happens after the \acrshort{rrc} timer and requires two-way communication which is presented in the tests. The device acknowledges the release and goes into \acrshort{rrc} idle mode.

\subsection{\acrshort{rrc} idle mode}
After ending \acrshort{rrc} connected mode the device enters \acrshort{rrc} idle mode listening for paging, using approximately 1\acrshort{ma}. The default \acrfull{T3324} timer is 10 seconds and in this mode the device can receive data from the network with a paging request, meaning that there is data for the device in the network. If there is more data in the network the device will re enter \acrshort{rrc} connected mode and try to receive data from the network. If there is no more data for the device it can enter \acrshort{psm} or \acrshort{edrx}. The tests showed the device was only in \acrshort{rrc} idle mode for approximately one second before entering \acrshort{edrx}, followed by \acrshort{psm} mode.

\subsection[\acrlong{edrx}]{\acrfull{edrx}} \label{ssection:edrx}
\acrfull{drx} is a feature in \acrshort{lte} to keep the device connected, but reducing the power usage by only checking for new data (paging) on time slots. After a sequence of communication the device goes into a \acrshort{drx} state where it periodically checks if the network has new information for the device. If there is information for the device, the device enters \acrshort{rrc} connected mode where it can receive and send data.

When designing \acrshort{nb-iot}, \acrshort{drx} had to be improved. However, if the device always checks for paging the battery level would decrease at a high rate. \acrshort{edrx} is therefor an outcome of the idea of \acrshort{drx} in the special case of sensor networks. It is a new way to handle communication for sensors which needs to wake up to certain events or messages from the network. I have included the figure \vref{figure:edrx}, illustrating \acrshort{edrx} from an article about \acrshort{edrx} and \acrshort{psm} for \acrshort{lte-m1}. Even though this is for \acrshort{lte-m1} it applies for \acrshort{nb-iot} as well. After a device has communicated, either sending or receiving packets over the network, it is desired that the \acrshort{ue} use as little power as possible. By default, the device will enter \acrshort{drx} mode for a network-specified time and if configured go into \acrshort{psm} mode after the given time.

By enabling \acrshort{edrx}, the device will enter \acrshort{edrx} after the \acrshort{rrc} idle period. This means that for most of the time the device will sleep, but periodically it will check for new information (paging). The period where the device sleeps is configurable in the network. Since the internal clocks in the devices are unreliable, and the time between these syncs may be long (typically 1-2 hours), the \acrshort{ue} and network need to first sync the clock and information between them. This operation is called sync guard and is needed due to communication based on time (time slots/\acrlong{tdma}). After this process the device will operate in normal \acrshort{drx} mode for a certain period (\textasciitilde10 seconds) and go to sleep after that period. This is a very power efficient way to keep the device more synchronized with the network.

\begin{figure}[ht]
  \centering\includegraphics[width=12cm,height=6cm]{/Users/henninghakonsen/Dropbox/Masteroppgave/thesis/latex/images/edrx.jpg}
  \caption[\acrshort{edrx} operation]{\acrshort{edrx} operation \cite{online:edrxpsm}}
  \label{figure:edrx}
\end{figure}

\acrshort{edrx} has a set of configurable parameters which define how the \acrshort{ue} behave when entering \acrshort{edrx} mode. In short terms this means specifying \acrshort{edrx} cycle length and paging time window. The cycle length is the length of an \acrshort{edrx} cycle. This means the time from start of a period to the start of the next period. The cycle length can vary from 5 to X seconds. The paging time window can be set to 0 to 20 seconds, where the value 0 means that paging is not in use.

%%http://www.link-labs.com/blog/lte-e-drx-psm-explained-for-lte-m1
\subsection{Paging} \label{ssection:paging}
General paging is a procedure used for communication between \acrshort{ue} and \acrshort{enb}. It is the first step for initiating communication and if paging is enabled the \acrshort{ue} will listen for data from the network at given time intervals within a \acrshort{rrc} or \acrshort{edrx} period. Today paging is allocated on what is called the anchor \acrshort{prb}, hence this resource can be heavily loaded. The paper, "Investigation About the Paging Resource Allocation
in NB-IoT"\cite{IEEEXplo82:online}, has tried to improve paging for \acrshort{nb-iot} and they propose to utilize several resource blocks, and their simulations shows an 30.5\% improvement in resource utilization and around 80\% reduce in power consumption.

\subsection[\acrlong{psm}]{\acrfull{psm}} \label{ssection:psm}
\acrshort{psm} is the core of power saving in \acrshort{nb-iot}, as well as many other \acrshort{lpwan}. The idea of almost all \acrshort{lpwan} is that the devices send information periodically, while sleeping the rest of the time. After a typical connected period the \acrshort{ue} want's to enter sleeping mode where very little power is drawn, only by an internal clock. This mode is called \acrshort{psm} and enables the device to sleep for a network configured time period. In the test case this time (\acrfull{T3412}) was set to 300 hours, almost two weeks. That means that the device will be removed from all metadata in the \acrshort{mme} after 300 hours if there is no communication. The way it works is that the device goes from connected mode, to \acrshort{drx} mode listening for information from the network. After this the device enters \acrshort{psm} mode if activated. When in \acrshort{psm} mode the only thing going on in the device is an internal clock which can start the system at a specific time. This is necessary since the device has to wake up and transmit uplink data at certain user defined intervals.

In \acrshort{psm} the power usage will be as low as 3$\mu$A, which is extremely low and the main reason why one can expect these types of sensors to achieve 10 years of battery lifetime.

\subsection{\acrlong{ecl}} \label{ssection:ecl}
The \acrshort{ue}s will enter different coverage modes based on their signal strength towards the \acrshort{enb}. There are three values of \acrfull{ecl}. At which signal power level devices switch over to the different levels is decided by the network, e.g. ${-105 \acrshort{dbm}}$ for \acrshort{ecl} 1 and ${-115 \acrshort{dbm}}$ for \acrshort{ecl} 2, signal power better than this resolves to \acrshort{ecl} 0.

\acrshort{ecl} is not important by itself, but an application fetching the \acrshort{ecl} of the device will gain a lot of information about the reception, hence it might reconfigure the setup of the transmit process for example. Based on the signal strength the device will retransmit data to improve \acrfull{tsr} towards the application server. In \acrshort{ecl} 0 the device will transmit the data once, and in level 1 and 2 the \acrshort{ue} will retransmit the data to ensure delivery of the packet. Section \vref{ssection:ecltest}, gives insight into the transmit process of two transmits, one with \acrshort{ecl} 0 and one with \acrshort{ecl} 2. The difference is big and will definitely affect the battery lifetime of the device.

\subsection{The transmit procedure} \label{ssection:transmitprocedure}
We have looked at several techniques for \acrshort{nb-iot} and the specifications looks very good when the \acrshort{ue} has good coverage. In good coverage areas devices will transmit data with a transmit power of ${-40 \acrshort{dbm}}$, with an average actual transmit time of \textasciitilde200ms. Using the calculated values from section \vref{ssection:prerequisites}, this resolves to a power usage of ${((244 \acrshort{mw} / 60 / 60) * 0.2s = ~0.0135 \acrshort{mwh}}$. However, in worst case situations, the sensor will boost the signal up to ${+23 \acrshort{dbm}}$, and as stated, a sensor with bad reception will try to retransmit data. Depending on the signal power the device will enter different coverage modes, as previously described in section \vref{ssection:ecl}. Given that the sensor has bad coverage, and is operating in \acrshort{ecl} 2 the transmit time might be as long as five seconds for a single data transmit.
The actual power usage increases rapidly and the chip would use 726\acrshort{mw} for 5 seconds, resulting in a power usage of ${((726 \acrshort{mwh} / 60 / 60) * 5s = ~1.01 \acrshort{mwh}}$, which is 74 times increase in power usage compared to optimal conditions. However, as revealed in the tests, ${-40 \acrshort{dbm}}$ is normally not possible to achieve.

The following figure \vref{figure:ecl_dbm}, presents an outline of how the device chooses \acrshort{ecl} mode dependent on the signal quality. In theory the device should lower its transmit power equal to the signal strength. The graph represents the theory behind the adjustment of the transmit power and you will see how this actually works in section \vref{section:testing}.

%%https://cdn.rohde-schwarz.com/pws/dl_downloads/dl_application/application_notes/1ma266/1MA266_0e_NB_IoT.pdf

\begin{figure}[H]
  \centering\includegraphics[width=\textwidth,height=9cm]{/Users/henninghakonsen/Dropbox/Masteroppgave/thesis/latex/images/ecl_tx_power_relation.png}
  \caption[\acrshort{ecl} and transmit power relation]{Relation between \acrshort{ecl} and transmit power}
  \label{figure:ecl_dbm}
\end{figure}

\section{The current market and deployment strategies} \label{section:market}
I have presented several applications which suit \acrshort{lpwan} and there is a demand for chips which implement \acrshort{nb-iot} or \acrshort{lte-m1}. This section gives a brief introduction to the key firms involved with \acrshort{lpwan} and how the market looks like in early 2018.

\subsection{Manufacturers and stakeholders} \label{ssection:manufacture}
There are many companies which manufacture hardware for the mobile industry. Historically Neul (UK), HiSilicon (China) and Qualcomm (USA) has been the main chipset producers. Neul and HiSilicon are owned by Huawei which makes them one of the biggest producers at the moment. In addition, there are three companies, Ublox, Telit and Quecktel, which have been using "Huawei" chipsets, only adding their custom software. At the moment the agreement towards Huawei has expired and the three software companies will need to source hardware from another manufacturer, which will most likely be Qualcomm.

Currently, there is a problem surfacing in USA which stresses the market. In Europe, GSM/GPRS has been the only alternative, while in the U.S. there has been three network technologies, GSM/GPRS, IS95 and TDMA. Because of the fragmented market, the process of developing chipsets has been difficult and they have now decided to take down IS95 and TDMA. Most users have gone over to LTE, but there are many companies which use IS95 and TDMA for \acrshort{m2m} communication which now has no good alternative. The solution is to move over to another \acrshort{lpwan}, which for most companies will resolve to \acrshort{lte-m1} because of the opportunities and spread of applications this technology provides.
Because of the stress in the market and the increased demand for general \acrshort{lpwan}, all chipset manufacturers began their work on \acrshort{nb-iot} and \acrshort{lte-m1} chipsets. Most of the companies did however reuse their \acrshort{lte} chips to produce \acrshort{lpwan} chipsets. The chip used for my tests is produced by Neul, with software from Ublox, and suffers from the process of reusing hardware originally developed for \acrshort{lte}. Section \vref{section:thechip}, gives a detailed introduction to how I used the chip, and it is worth noting that the chip was not production ready.

Other companies have seen the development of this market and started to take up the fight against the biggest companies. Nordic Semiconductor is one of these companies, and originally has produced chips for other technologies like Bluetooth. According to a talk they gave, they have used 1.000.000 working hours on chips with \acrshort{lte-m1} and \acrshort{nb-iot}, which is developed from scratch focusing on \acrshort{lpwan} technology. They have seen the stressed market in the U.S. and will begin production of \acrshort{lte-m1} chips this year, followed by \acrshort{nb-iot} in 2019 \cite{person:ola}.

Based on the research done in this thesis it will be interesting to see the next generation \acrshort{nb-iot} chipsets. These will hopefully be more stable with an even bigger emphasis on power management. The tests performed revealed very promising results given the prerequisites.

\section{Challenges} \label{section:challenges}
With the kind of bandwidth we have today there is less concern with the amount of data transmitted. \acrshort{nb-iot} introduces developers to a new kind of applications which should be provident and fully operate for potentially ten years - this introduces difficult programmatic problems. I stumbled upon some of these issues and have included a set of best practice guidelines based on the experience revealed from the tests in section \vref{section:guidelines}.

The following part of the thesis will cover the project related to the thesis - the testing phase. This part was obviously the most demanding and I have tried to stress the network in ways not intended. Moreover, because of the pre-release of hardware and software[\ref{ssection:manufacture}] I need to state that the results, especially for the long-term test, should be considered as first results and will change as hardware and software evolves. I will test both Telenor and Telia as they provide \acrshort{nb-iot} in Oslo, and it is important to note that I am trying to give an objective statement about \acrshort{nb-iot} and not compare the two network providers.

\chapter{Where, why and how?} \label{chapter:wherewhynhow}
To get accurate and meaningful results there was a long planning phase. First of all I needed to decide how and what to test, and decided to set up a server which can receive data from a \acrshort{nb-iot} \acrshort{ue}. In addition, this server will be able to analyze the data and display the results in a meaningful way. I was so lucky to borrow a development kit for \acrshort{nb-iot} from Q-Free which enabled me to connect to the network and transmit uplink packets towards the server. I received SIM-cards from Telenor and Telia, which ment that I could do some comparison between the two network providers. With this in mind, the next sections describe and introduce the different parts of the setup and show how they are tied together.

\section{Testing environment}
The project was performed in Oslo and the tests were executed at several locations - UiO, my apartment at Lambertseter (Avstikkeren 7) and at Q-Free's office by Solli Plass. With three locations and two network providers I could collect more information, hence the results got better.
Not all base stations in these areas are \acrshort{nb-iot} enabled, but you will see that different signal power levels will affect the results. The signal power of the chip is related to the supplied antenna and the direction of this. With the \acrshort{nb-iot} development kit there was an antenna and I could position it the direction with best result. See picture \vref{pic:uio_nbiot_setup}, to see the setup at UiO. In section \vref{ssection:mapsndistances}, you can see the distances between the different test sites. Unfortunately I was not been able to get the location of Telenor's base station at UiO, but it is likely that it is located at the same place since there are many cells at the location of Telia's cells.

\begin{figure}[H]
  \centering\includegraphics[width=12cm,height=8cm]{/Users/henninghakonsen/Dropbox/Masteroppgave/thesis/latex/images/uio_nbiot_setup.jpg}
  \caption{\acrshort{nb-iot} lab setup}
  \label{pic:uio_nbiot_setup}
\end{figure}

\subsection{Maps and distances} \label{ssection:mapsndistances}
\begin{figure}[H]
  \centering\includegraphics[width=\textwidth,height=0.4\textheight]{/Users/henninghakonsen/Dropbox/Masteroppgave/thesis/latex/images/map_uio}
  \caption[Distance map - IFI, UiO]{Device distance map over IFI, UiO. Referring to \url{www.finnsenderen.no}\cite{online:finnsenderen}, I believe Telenor's cells are located close to Telia's, but have not been able to confirm this by Telenor.}
  \label{figure:map_uio}
\end{figure}

\begin{figure}[H]
  \centering\includegraphics[width=\textwidth,height=0.4\textheight]{/Users/henninghakonsen/Dropbox/Masteroppgave/thesis/latex/images/map_qfree}
  \caption[Distance map - Q-Free]{Device distance map over Q-Free. The distances are similar, but there are more obstacles between Q-Free and Telia's \acrshort{enb} which affects the reception.}
  \label{figure:map_qfree}
\end{figure}

\begin{figure}[H]
  \centering\includegraphics[width=\textwidth,height=0.4\textheight]{/Users/henninghakonsen/Dropbox/Masteroppgave/thesis/latex/images/map_avstikkeren}
  \caption[Distance map - Lambertseter]{Device distance map over Lambertseter, Avstikkeren. The device did not manage to connect to Telenor's network at this location.}
  \label{figure:map_avstikkeren}
\end{figure}

\section{Devices} \label{ssection:devices}
I have used a set of devices to achieve the results and the following subsections will introduce you to the devices used and why they were necessary for my tests.

\subsection{Server}
To host the web application I applied for a server at UH-IaaS, which hosts servers for university projects. I received a well suited server with the necessary specifications located in Bergen. The server was at times unstable due to power outages, but no results were damaged due to the downtime. Section \vref{chapter:webapp}, will give a detailed description of the web application.

\subsection{\acrshort{nb-iot} development kit}
Early in 2018, Q-Free received a development kit for \acrshort{nb-iot} from Ublox. The kit is equipped with a \acrshort{nb-iot} chip called SARA N211 from Neul and software from Ublox (see picture \vref{pic:nbiotdevkit}). The development kit enabled me to easily develop software for the tests which was crucial. The development kit was connected to a computer over USB and allowed data through the serial port of the chip. I used this setup for all tests, which includes packet size, signal power, latency and general behavior of the network. I also used this setup for longer tests which sends messages continuously with a more practical frequency to the server. The computer will run python programs to test the different features. In many cases the programs will also log test runs locally for detailed graphs of the behavior of the chip. The following subsection gives a detailed introduction to the chip used.

\begin{figure}[ht]
  \centering\includegraphics[width=12cm,height=8cm]{/Users/henninghakonsen/Dropbox/Masteroppgave/thesis/latex/images/nbiot_devkit.jpg}
  \caption[Closeup of Ublox \acrshort{nb-iot} development kit]{Ublox \acrshort{nb-iot} development kit. The picture is taken with both attenuators attached, which is a total of ${30\acrshort{dbm}}$ signal loss.}
  \label{pic:nbiotdevkit}
\end{figure}

\subsubsection{Sara N211 Ublox chip} \label{section:thechip}
Section \vref{ssection:sensoroutline}, presented Q-Free's parking sensor, which houses an identical \acrshort{nb-iot} Ublox chip which is used for my tests. This section will focus on the most essential modem commands for the \acrshort{nb-iot} chip. Modem commands are referred to as AT-commands and I will use this notation throughout this paper. As previously mentioned in section \vref{section:energysaving}, there are configurable timers in the network, as well as flags to keep in mind while developing software for low powered sensors. These parameters can be configured with AT-commands and has proven to give good results when taking power usage into consideration\footnote{I have used the default timers in this paper}. In addition I will show you the AT-command for sending data, and the structure of the data being sent.

\subsubsection{Monitoring - NUESTATS}
Monitoring is an important part of developing and a way to verify that software is running as optimal as possible. I wanted a way to monitor the performance of the \acrshort{nb-iot} chip, hence I choose to take advantage of the precise statistics given from the command \textbf{AT+NUESTATS}. According to the specifications of \acrshort{nb-iot}, there are a number of interesting thresholds and \textbf{AT+NUESTATS} will give me most of this information. In table \vref{table:nuestats}, you can see the output of the command.

\begin{table}[H]
\centering
\resizebox{\textwidth}{!}{%
\begin{tabular}{|l|m{10cm}|}
  \hline
  \textbf{Property} & \textbf{Description} \\ \hline
  Signal power & NB-IoT signal power expressed in tenth of dBm \\ \hline
  Total power & Total power within received bandwidth expressed in tenth of dBm \\ \hline
  Tx power & Transmit power expressed in tenth of dBm \\ \hline
  Tx time & Elapsed transmit time since last power on event expressed in milliseconds \\ \hline
  Rx time & Elapsed receive time since last power on event expressed in milliseconds \\ \hline
  Cell id & Physical ID of the \acrshort{enb} providing service to the \acrshort{ue} \\ \hline
  \acrshort{ecl} & Last \acrshort{ecl} value \\ \hline
  \acrshort{snr} & Last \acrfull{snr} value \\ \hline
  \acrshort{earfcn} & Last \acrfull{earfcn} value \\ \hline
  \acrshort{pci} & Last \acrfull{pci} value \\ \hline
  \acrshort{rsrq} & Last \acrfull{rsrq} value \\ \hline
\end{tabular}%
}
\caption{\textbf{NUESTATS} command}
\label{table:nuestats}
\end{table}

I believe that using \textbf{AT+NUESTATS} will give me accurate results, but I did stumple upon some problems which will be covered in section, \vref{ssection:ecl_255} and \vref{section:deviations}.

\subsubsection{Time - CCLK} \label{ssection:timecclk}
Time is another tool one often use for developing software. One would want to know how long a process endures, and in my case I wanted to monitor the latency of a send operation from the \acrshort{ue} to the server. The AT-command \textbf{AT+CCLK?} can read the time from the chip and this time is kept in sync with the network and stored in the \acrfull{mt}. The read operation returns a string with the time in the following format "yy/MM/dd,hh:mm:ss+TZ", where the characters represent, year, month, day, hours, minutes, seconds and time zone. However, I soon realised that this internal clock was not very good and after the device had been operational for a day the internal clock would be skewed and the latency results were not accurate. Section \vref{section:deviations}, will include a summary of this problem, but my suspicion is that the clock is not updated very often and with an imprecise internal clock the time will be wrong. As I were perfoming all tests with a computer connected to the development kit I started using the clock of the machine connected to the kit and send this timestamp to the server. This did not only provide a more updated clock, but the timestamp also included milliseconds which gave the result higher precision.

\subsubsection{\acrshort{edrx} settings - CEDRXS}
Section \vref{ssection:edrx}, explained how \acrshort{edrx} works and how one can configure the \acrshort{ue}. The AT-command \textbf{AT+CEDRXS} can be used to define how \acrshort{edrx} is performed on the \acrshort{ue}. One can for example issue this AT-command, \textbf{AT+CEDRXS: 1, 5, "0101"}, to put the \acrshort{ue} into \acrshort{edrx} mode with cycle length of 163.84 seconds\footnote{This is the \acrshort{nb-iot} default value for \acrshort{edrx}.}.

\subsubsection{\acrshort{psm} settings - CPSMS}
One can also define how \acrshort{psm} is used by the \acrshort{ue}. One can enable \acrshort{psm}, and set the timer \acrfull{T3412}\footnote{The \acrshort{nb-iot} default value of \acrshort{T3412} is 54 m.\cite{atcommand:ubloxchip}} with this command. If \acrshort{psm} is enabled, the device will enter deep sleep after expiry of the timer \acrfull{T3324}\footnote{The \acrshort{nb-iot} default value of \acrshort{T3324} is 60 s.}.

\subsubsection{Create socket - NSOCR}
The command opens a port on the \acrshort{ue}, which enables data transfer on the given port. My application opens an \acrshort{udp} socket on a port, and the port used can be a number between 0-65535, except for 5683 since this is used by the chip to send \acrshort{coap} messages. The command returns a socket number which will be used for subsequent send (NSOSTF) commands.

\subsubsection{Send command with flags - NSOSTF}
This AT-command is used to send \acrshort{udp} packets on a given port with a specified flag. The command takes a set of parameters and the data in hexadecimal format. The parameters are, socket number (retrieved from the AT-command NSOCR), remote IP address, remote port, flag, data length and actual data. The flag can be used to specify the behavior of the \acrshort{ue} after the data has been transmitted, see table \vref{table:flag} for details.

\begin{table}[H]
\centering
\resizebox{\textwidth}{!}{%
\begin{tabular}{|l|m{10cm}|}
  \hline
  \textbf{Mode} & \textbf{Description} \\ \hline
  0 & No flags are set \\ \hline
  1 & Send message with high priority \\ \hline
  2 & Indicates \acrshort{rai}. This mode will put the \acrshort{ue} into \acrshort{psm} mode directly after transmitting data. Meaning less time in power heavy modes. \\ \hline
  3 & Indicate release after next message has been replied to. \\ \hline
\end{tabular}%
}
\caption[\acrshort{nb-iot} send command flag options]{Transmit flag options}
\label{table:flag}
\end{table}

The data transmitted towards the server adds an additional \acrshort{coap} header followed by the actual data. I decided to do this to reduce logic on the server and create the environment similar to a real world application.

\subsubsection{Signaling connection status - CSCON}
One can verify the connection status of the \acrshort{ue} with the command \textbf{AT+CSCON?}. If the reply is 1 this means that the \acrshort{ue} is in \acrshort{rrc} connected mode and is able to send and receive packets. If the reply is 0 this means that the \acrshort{ue} is in idle mode, hence the device is inactive. Note that this does not mean that the device is in \acrshort{psm} mode. This only occours after the configured \acrshort{T3324} timer expires.

\subsubsection{Network registration status - CEREG}
The command AT+CEREG can verify the network registration status of the \acrshort{ue}. This command is a very useful for any application which wants to check if the \acrshort{ue} is still connected to the network.

\subsubsection{\acrshort{psm} status - NPSMR}
Gives the status of \acrshort{psm} on the device. It will return \textbf{1,1} if the device is in \acrshort{psm} mode which means that the power usage is very low.

\subsection{Fluke precision multimeter}
Q-Free owns a precision multimeter which I used to measure the current through the \acrshort{nb-iot} chip. With the multimeter I managed to pinpoint different stages of the chip and by adding corresponding statistics from \textbf{AT+NUESTATS} the performance of the results increased. As sketched in figure \vref{figure:labsetup}, the multimeter was connected to the computer using an ethernet cable, enabling readings from the device.
Depending on the selected precision of the measurements the device collected between 30-300 samples per second. The most detailed tests required high sample rate, while the longer tests used lower precision measurements. The device responded on commands much like a \acrfull{repl}, where you issue a request with a command and the device will respond with an answer. In table \vref{table:fluke_commands}, you can see the commands I used and a short description related to them.

\begin{table}[H]
\centering
\resizebox{\textwidth}{!}{%
\begin{tabular}{|l|m{10cm}|}
  \hline
  \textbf{Command} & \textbf{Description} \\ \hline
  :INIT & Clears old readings \\ \hline
  :FETCH? & Fetches all readings from the device. The response is a string with numbers divided with "," \\ \hline
\end{tabular}%
}
\caption{Fluke commands}
\label{table:fluke_commands}
\end{table}

\section{Software}
I developed many test programs with a specific setup (see figure \vref{figure:labsetup}, for an overview) as well as a web application to give me the knowledge and experience to give a precise report about the status of \acrshort{nb-iot} the spring of 2018. A thing I learned during the testing phase is that the output format often changed. Since the programs produces results based on data which is already processed it meant that some tests had to be reperformed. In hindsight it would be better to capture the raw data and develop a program which generated graphs based on the data. However, I am satisfied with the end result and you will see many of the generated graphs in section \vref{section:testing}.

\begin{figure}[ht]
  \centering\includegraphics[width=10cm,height=10cm]{/Users/henninghakonsen/Dropbox/Masteroppgave/thesis/latex/images/labsetup.jpg}
  \caption[Lab setup overview]{Lab setup \cite{pcpng35052:online} \cite{fluke88435:online} \cite{ingagrue31:online}}
  \label{figure:labsetup}
\end{figure}

I started experimenting with different AT commands and created small programs to test the functionality of the chip and the connection towards the server. I created a simple program which sent \textbf{AT+NUESTATS} to the application server and went on transforming this program to helper methods which would be included in later test programs. A sample packet with a \acrshort{coap} header would look something like the example \vref{samplecommand}, where the payload is converted to hexadecimal. I used some time creating the correct \acrshort{coap} packet for the server to accept it as a \acrshort{coap} packet. If you don't want to use \acrshort{coap} you can simply fill the payload with data without the \acrshort{coap} header, but this means that you are missing the features of \acrshort{coap} which could be beneficial for your application. Dropping the \acrshort{coap} header means that you are sending a pure \acrshort{udp} packet through the network and it is harder to verify the packet when it is received on an application server.

\begin{lstlisting}[caption={\acrshort{nb-iot} sample transmit},label={samplecommand},language=Python]
  AT+NSOSTF=0,"158.39.77.97",5683,0x0,88,"500230
  30B131112AFF2D313039365F2D313034325F3233305F39
  3732315F37363838305F33343433393435325F315F3132
  335F363235325F39385F2D3130385F31382F30322F3139
  2C31363A31363A35362B30305F31325F"
\end{lstlisting}

I proceeded working on communication towards the multimeter, which was quite easy. The multimeter was given an \acrshort{ip} address which was related to the \acrshort{ip} address of the connected machine. Creating a socket to this IP address in python and connecting to it was simple. The test programs starts off by clearing all buffers with \textbf{:INIT} and then fetch data on a regular interval with \textbf{:FETCH?; :INIT}. Each reading was converted into a list of numbers which was appended to a global list of readings. Fetching for new data every 5th second was a good tradeoff between fetching all the time and very rare. These functions were added to the helper file \textbf{nbiot\_labtest\_helpers.py}, and forward I went on writing a couple of programs to monitor the behavior of the chip. The following subsection includes documentation on the most essential test programs.

\subsection{Test programs}
\paragraph{\textbf{at\_command\_test.py}} is a program well suited for testing different AT commands which is necessary to explore and research how the device handles different features. The program is like a \acrshort{repl} and requests an optional message and a command. If a message is supplied the program assumes that the request is a send command. If not, the command is handled and the respons is printed in the terminal. Note that this program is written in python V2 while the other programs expect python V3 to execute. See \href{https://github.com/henninghaakonsen/thesis/blob/master/code/at_command_test.py}{github}\cite{code:atcommand} for complete code.

\paragraph{\textbf{nbiot\_labtest.py}} continuously transmits packets over \acrshort{nb-iot} - either status packets towards the server or random data of different sizes. It enabled me to test different scenarios and log the results in a representable way with graphs. The program takes in a list of parameters described in table \vref{table:labtestparameters}.

\begin{table}[H]
\centering
\resizebox{\textwidth}{!}{%
\begin{tabular}{|l|m{10cm}|}
  \hline
  \textbf{Parameter} & \textbf{Description} \\ \hline
  -id & ID which you want to POST towards the server. Default: 0 \\ \hline
  -gn & Output graph name. The supplied name will lead the heading, followed by a list of the rest of the parameters. \\ \hline
  -d & Set the desired delay between each iteration. Default: 5 \\ \hline
  -i & Set the desired iterations. Default: -1 which means that the program will loop until the user presses ctrl+c \\ \hline
  -b & Set the desired length of the payload in bytes. If 0 is requested, statistics from \textbf{NUESTATS} will be sent to server. Default: 100 \\ \hline
  -r & Set \acrshort{rai}. Default: false \\ \hline
  -l & Set device logging. Default: false \\ \hline
  -rb & Set test to reboot \acrshort{nb-iot} chip directly after execution. Default: false \\ \hline
\end{tabular}%
}
\caption[\textbf{nbiot\_labtest.py} parameters]{\textbf{nbiot\_labtest.py} parameters. See \href{https://github.com/henninghaakonsen/thesis/blob/master/code/nbiot_labtest.py}{\acrshort{nb-iot} labtest}\cite{code:nbiotlabtest} for complete code}
\label{table:labtestparameters}
\end{table}

To get more stable tests and to automate the process I added logic to perform a set of tests when starting the program. If byte size is not given to the program the execution will perform 4 tests on 50, 100, 200 and 512 bytes. Those 4 tests are a combination of logging and \acrshort{rai}.

For each iteration the program sends a message towards the server, followed by a period of logging. If logging is enabled the program fetches information from the chip with \textbf{NUESTATS} and three other commands - \textbf{AT+NPSMR?}, \textbf{AT+CSCON?} and \textbf{AT+CEREG?}. From \textbf{NUESTATS} the program retrieves information about transmit and receive time, transmit power, \acrshort{ecl} and signal power. I choose these parameters because they give good information by themselves, as well as they combined gives a good indication of what the chip is doing at any time. These results were logged approximately four times per second - limited to the speed of the serial bus between the computer and the development kit. Every fifth second it also fetched stored information from the precision multimeter. Depending on the configured precision of the multimeter, 50-400 samples were logged per second which was correlated to the logged information from the chip. With all results the program produces a set of graphs which can be used for analysis of the network and the chip.

Because there were no way to retrieve the timestamp of each measurement of the multimeter I had to flatten the result and apply a timestamp to each measurement. This means that I assume that the interval is equal. When looking at the figures in section \vref{section:detailedtest}, the power graphs are mostly alligned with the results created from \textbf{NUESTATS}, but often one or two seconds in front of the other graphs.

\textbf{Graphs generated:}
\begin{itemize}
  \item Receive: Shows the percentage of how much time the device were receiving compared to the actual interval.
  \item Transmit: Shows the percentage of how much time the device were transmitting compared to the actual interval. Both receive and transmit scales use 1 as 100\%.
  \item Transmit power: Shows the transmit power of the device.
  \item Signal power: Shows the signal power of the device.
  \item \acrshort{ecl}: Shows which signal power level the device is currently in.
  \item \acrshort{psm}: Shows the status of \acrshort{psm}.
  \item \acrshort{rrc} state: Shows the \acrshort{rrc} state.
  \item Registration status: Shows the registration status of the chip. 1 is connected, 4 is unknown and 5 is searching for network.
  \item Power usage: Shows the current consumption of the chip in \acrshort{ma}.
\end{itemize}

Based on the average current of one test the power usage can be calculated. The average current can be converted to \acrshort{mwh} and convert this to \acrshort{mws} and multiply the result with the duration of the test. This is an example from \href{http://158.39.77.97:9000/\#/results/UiO\_TELIA\_5.02\_precision\_2018-03-16\_1\_0x2\_60\_1\_512}{webapp} \cite{webapp:powerexample}: ${( (9.2089 \acrshort{ma} * 3.3V) / 60 / 60 ) * 60 = ~0.5064 \acrshort{mwh}}$. With this information it is possible to show the difference in power usage with different scenarios and use the results to outline the lifetime of a device under different kinds of environments. The constant \acrshort{ue} logging with \textbf{NUESTATS} consumes a lot of power, hence the program can be started without \textbf{NUESTATS} (do not use '-l') logging, which gives a better overview of the actual power usage. Most of the tests are performed both with and without the '-l' parameter.

\paragraph{\textbf{power\_calculator.py}} calculates the power consumption over a given period from one of the test results from the short-term tests. This tool was important in the process of figuring out how much power specific parts of the transmit uses. It extracts all power measurements in \acrshort{ma}, between two timestamps and calculates the average current, multiplying this with the time delta between the two timestamps supplied. The program takes in a list of parameters described in table \vref{table:powercalculator}.

\begin{table}[H]
\centering
\resizebox{\textwidth}{!}{%
\begin{tabular}{|l|m{10cm}|}
  \hline
  \textbf{Parameter} & \textbf{Description} \\ \hline
  -f & File name. Must be from short-term test and have .html format \\ \hline
  -s & Start time \\ \hline
  -e & End time \\ \hline
\end{tabular}%
}
\caption[\textbf{power\_calculator.py} parameters]{\textbf{power\_calculator.py} parameters. See \href{https://github.com/henninghaakonsen/thesis/blob/master/code/power_calculator.py}{power calculator}\cite{code:powercalc} for complete code}
\label{table:powercalculator}
\end{table}

\paragraph{\textbf{nbiot\_labtest\_details.py}} tests different packet sizes. The program transmits 6 packets with different packet sizes\footnote{25, 50, 100, 200, 400 and 512 bytes} a given number of times. After the given iterations are processed the program produces two graphs, one with all recordings and one with normalized data\footnote{I have choosen to include average, sum, min and max values for each packet size}. The program takes in a list of parameters described in table \vref{table:nbiotdetails}.

\begin{table}[H]
\centering
\resizebox{\textwidth}{!}{%
\begin{tabular}{|l|m{10cm}|}
  \hline
  \textbf{Parameter} & \textbf{Description} \\ \hline
  -gn & Output graph name. The supplied name will lead the heading, followed by a list of the rest of the parameters. \\ \hline
  -d & Set the desired delay between each iteration. Default: 5 \\ \hline
  -i & Set the desired iterations. \\ \hline
  -r & Set \acrshort{rai}. Default: false \\ \hline
\end{tabular}%
}
\caption[\textbf{nbiot\_labtest\_details.py} parameters]{\textbf{nbiot\_labtest\_details.py} parameters. See \href{https://github.com/henninghaakonsen/thesis/blob/master/code/nbiot_labtest_details.py}{\acrshort{nb-iot} details}\cite{code:nbiotdetails} for complete code}
\label{table:nbiotdetails}
\end{table}

\chapter{The web application} \label{chapter:webapp}
The long-term tests generated a lot of data, and the simple solution would be to collect the data and import it into Excel for analysis. However, there are approximately eighteen thousand elements in the database, thus maintaining and keeping track of this data in Excel would not suffice for an effective analyzation of the data after the test. I decided to implement a web app to display information about the sensors. This web app can display latency, signal power and other power related data about each sensor. Most of the practical knowledge was gained from results coming into the web application and I would not perform the kind of short-term tests which gave the most interesting results without the web application. I have included some results from the long-term tests in the paper, but if you would like to explore more results I recommend looking at Q-Free, Telenor 09-10 March 2018 and Avstikkeren, Telia 13-15 March 2018\footnote{Here you can see that the latency is affected by the clock bug we discussed in section \vref{ssection:timecclk}}.

The following sections discusses the implementation of the server and the web app.

\section{Backend: Node.js}
I decided to use node.js for what would become the backend of the server. Node.js is a new and modern way to create simple REST APIs. A REST API follows a set of rules to make it robust and stateless. It also should include all CRUD operations (create, read, update and delete\footnote{The server only implements create and read}) and should use the appropriate request method, such as POST, GET, PUT and DELETE. A big difference from standard backend programming is that Node.js uses javascript as programming language. Even though javascript is not the cleanest programming language, it is efficient and the codebase is kept low. In addition node.js takes advantage of using \acrfull{npm}, which offers packages for many kinds of applications, both frontend and backend.

The server hosts multiple API endpoints and the most important are collecting data and retrieving data. The database contains a lot of data collected from different tests and it would require a lot of rendering to do analysis on the client side. I decided to analyze the data at specific intervals so that the client would simply display the data. The analysis and data is described in section \vref{ssection:analysis}.

\subsection{The database}
There are multiple ways to store data and there is always a trade off when choosing the platform. PostgreSQL is a great example of a database which is fast and reliable, but it can be complicated to set up and it is SQL driven which means all data has to fit into a pre-defined configuration. Since this project is small and the data could potentially change during the test phase, I chose to use a NoSQL database. There are many good tools for storing data in a NoSQL database. MongoDB is the most commonly used and works well with Node.js, hence I decided to use it for data storage. Subsection \vref{subparagraph:mongodb}, explains how mongoDB was used in my implementation.

\section{JavaScript packages for the server}
The web app uses different packages from \acrshort{npm} to include certain features. By including quality packages the server logic and size was kept low and I could focus on the data. The following subsections gives a short introduction to the most important packages used.

\subsection{Express}
Express is a fast and robust package for handling requests and offers nice APIs for listening on specific ports and so forth\footnote{See \cite{npm:express} for documentation}. The code example \ref{code:express}, shows the base setup for making a server handle requests. The port is set with a combination of the processes port and user specific port selection. The reason why this might be useful is if your server is hosted on a payed server which might be deployed on different IP and port for each deploy. With this implementation you won't need to consider this. In the example you can see a GET for a specific path, however you can also redirect all routes to a separate file for cleaner code. The last thing one need to do is to listen for incoming requests on the specific port. Now express will handle all request with a single threaded event loop.

\begin{lstlisting}[caption={Base express setup},label={code:express},language=JavaScript]
// express setup
const server = express();
const port = process.env.PORT || 8020;

server.get('/', function (req, res) {
  res.send('Hello World')
})

server.listen(port, () => {
  console.log(`app started on`, port);
});
\end{lstlisting}

\subsection{\acrshort{coap}}
\acrfull{coap} is a simple network protocol and offers requests like HTTP, but without all the overhead. \acrshort{coap} is designed for low powered devices and will be used for testing. I have used node-coap package for including \acrshort{coap} support on the server\footnote{See \cite{npm:coap} for documentation}. Since the protocol is simple, the requests needs to be handled a bit different. Code example \vref{code:coap}, shows a simple node-coap setup.

\begin{lstlisting}[caption={Base \acrshort{coap} setup},label={code:coap},language=JavaScript]
var coap = require('coap')
var coap_server = coap.createServer()

coap_server.listen(port, () => {
    logger.log("info", `Worker ${process.pid} started coap server on ` + port);
})

// All request is handled by this function. It is not possible to request a specific url path
coap_server.on('request', function(req, res) {
    // Payload of the request is a byte stream. Parse the stream and handle the data
    var data = JSON.parse(req.payload.toString());
})
\end{lstlisting}

\subsection{Cluster}
With express all request will be handled by one thread as Node.js is single threaded. Since the server basically puts everything on hold while analyzing the data, there was a need for a way to handle request at the same time. The package cluster can handle several requests asynchronous and is based on web workers which are an abstraction for processes in Node.js\cite{npm:cluster}.

The code example \vref{code:cluster}, shows how to enable multiple processes to handle requests to improve efficiency. The cluster package offers a set of functions, so the first thing to do is to verify which worker is master. The master then forks a number of workers which will be divided into processing one specific task, either analysis or requesting \acrshort{http}/\acrshort{coap} requests. Since the server will run over a long time it is important to handle workers which die. The master process will pick up dead workers and respawn the worker so that the server can continue processing requests. It is very important that the worker is forked with the same id since the worker depends on the id to select which task to perform.

\begin{lstlisting}[caption={Express setup with cluster},label={code:cluster},language=JavaScript]
const server = express();
const coap_server = coap.createServer()

const cluster = require('cluster');
const numCPUs = require('os').cpus().length;

if (cluster.isMaster) {
    // Fork workers.
    for (let i = 0; i < numCPUs; i++) {
      const worker = cluster.fork();
      notify worker with id
    }

    // Respawn workers on exit
    cluster.on('exit', (worker, code, signal) => {
      const new_worker = cluster.fork();
      notify worker with id

      // Remove old entry from map
      delete worker_map[worker.id]
    });
  } else {
    // Send message to master process to get appropriate id
    process.send({ msgFromWorker: '' })

    // Receive message from the master process.
    process.on('message', function (msg) {
      let id = msg.id;

      if (id == 1 and numCPUs > 1) {
        // Start analysis task
      } else if (id == 2) {
        server.listen(port, () => {
        });
      } else {
        coap_server.listen(() => {
          // Start coap server
        })
      }
    });
  }
\end{lstlisting}

\subsection{MongoDB} \label{subparagraph:mongodb}
Storing data is key in any server and the server use the MongoDB package. The package includes a nice API for MongoDB storage and is simple, and highly effective\cite{npm:mongodb}. MongoDB can consists of several databases, and each database can contain several collections. An entry within a collection is called a document, and the document can contain data with any size and structure.

One has to install MongoDB on the particular instance and on linux it is as simple as "sudo apt get mongodb". The install process creates a service, mongodb.service, and this service starts at boot time and is ready for incoming connections.

The code example \vref{code:mongodb}, shows how to connect to the db and insert a simple document in a collection with an API call.

\begin{lstlisting}[caption={MongoDB setup and insertion},label={code:mongodb},language=JavaScript]
const MongoClient = require('mongodb').MongoClient;

MongoClient.connect('mongodb://localhost:27017/db', (err, db) => {
    if (err) return err

    server.post(api + '/nodes', (req, res) => {
        const data = req.body;
        db.collection('nodes').insert(data, (err, result) => {
          if (err) {
            logger.log("error", "insert failed: " + err);
            res.send({ 'error': 'An error has occurred' });
          } else {
            res.send(result.ops[0]);
          }
        });
    });
}
\end{lstlisting}

\subsection{Moment}
Storing data with MongoDB is easy, but deciding the format of the contents can be difficult. A normal problem on server applications is server time, versus client/sensor time. A common approach is to use UTC time everywhere and the package moment is a great tool to keep track of time\footnote{See \cite{npm:moment} for documentation}. The code \vref{code:moment}, takes in a timestamp and creates a moment object in UTC time. This original timestamp is not converted, so the sensor also needs to send the timestamp as UTC time. The conversion of timestamps happens in the web application.

\begin{lstlisting}[caption={Simple moment example},label={code:moment},language=JavaScript]
server.post(api + '/nodes/:id', (req, res) => {
    let data = req.body;
    let timestamp = moment.utc(data.timestamp);
}
\end{lstlisting}

\newpage
\section{Frontend}
The web app is used for hosting data related to long-term tests and how the device performs on a more abstract level than with the detailed tests in section \vref{section:detailedtest}. The home page contains a short paragraph about the thesis and links to the latests version of the thesis, source code related to the project and results from the detailed tests, see screenshot \vref{pic:homepage}. Further more you can select nodes on the left which includes data from different sites and network providers, see screenshot \vref{pic:nodepage1} and \vref{pic:nodepage2}.

With a solid backend in Node.js I choose React for the web app. I will not go in detail on the different packages used for the web app, since this is not relevant to the thesis. However, the following section gives a brief introduction on how to use the app and what kind of analysis you can view with it.

\subsection{Layout} \label{sssection:layout}
By default the graphs will display data one day backwards from the latest data entry of the selected node. This means that if the day you visit the page is 20.07.18, and the latest data entry is 02.03.18, you will see data from 1-2 March of 2018. You can also inspect certain areas by selecting an area of one of the graphs and the other graphs will adjust.

\begin{figure}[H]
  \centering
  \includegraphics[width=\textwidth,height=9cm]{/Users/henninghakonsen/Dropbox/Masteroppgave/thesis/latex/images/homepage.png}
  \caption[SensorApp homepage]{SensorApp homepage. Additional material are linked at this page, and you can select a node on the left to explore the long-term results.}
  \label{pic:homepage}
\end{figure}

\begin{figure}[H]
  \centering
  \includegraphics[width=\textwidth,height=9cm]{/Users/henninghakonsen/Dropbox/Masteroppgave/thesis/latex/images/nodepage_1.png}
  \caption[SensorApp nodepage part 1]{SensorApp nodepage part 1. The figure displays the signal power and latency graph from Q-Free, Telenor 09-10 March, with very good results.}
  \label{pic:nodepage1}
\end{figure}

\begin{figure}[H]
  \centering
  \includegraphics[width=\textwidth,height=9cm]{/Users/henninghakonsen/Dropbox/Masteroppgave/thesis/latex/images/nodepage_2.png}
  \caption[SensorApp nodepage part 2]{SensorApp nodepage part 2. The figure displays the general statistics from Q-Free, Telenor 09-10 March, with very good results.}
  \label{pic:nodepage2}
\end{figure}

\subsection{Analysis} \label{ssection:analysis}
When having selected a node you can explore the different graphs, which is a good way to display statistical data. The graphs data represents the output of the analysis on the server and the following sections describe what the graphs present and why I choose the different aspects of the behavior of the network. Remember that the data is mostly produced by \textbf{AT+NUESTATS}, so read section \vref{section:deviations}, for any deviations.

\subsubsection{Signal power and Latency}
One of the main features of \acrshort{nb-iot} is coverage, hence displaying signal power statistics over time was important. One interesting point of signal power is that it is closely related to the power usage of the device. The latency of \acrshort{nb-iot} is not a key feature, but the desired latency is under ten seconds according to the specifications\cite{datasheet:ubloxchip} and by displaying the latency in the same graph as signal power it is clear how the different levels inflict latency in the network. The output of signal power and latency is directly from the output of \textbf{AT+NUESTATS}. To give a clearer understanding of signal power I have classified the values into categories, see table \vref{table:coverage_cat}.

\begin{table}[H]
\centering
\resizebox{\textwidth}{!}{%
\begin{tabular}{|l|m{10cm}|}
  \hline
  \textbf{Signal power (\acrshort{dbm})} & \textbf{Category} \\ \hline
  -40 to -60 & Excellent signal \\ \hline
  -60 to -80 & Good signal \\ \hline
  -80 to -100 & Quite good signal \\ \hline
  -100 to -110 & Bad signal \\ \hline
  -110 to -130 & Very bad signal \\ \hline
\end{tabular}%
}
\caption{Signal power categories}
\label{table:coverage_cat}
\end{table}

\subsubsection{Recieve and transmit time}
The output from \textbf{AT+NUESTATS} gives a receive and transmit counter in milliseconds. The value between each entry was used to calculate actual receive and transmit time. The analysis produces a value which represents how much time the device was in receive or transmit mode in one interval. This is also closely related to the signal power of the device and also the \acrshort{ecl} value which indicates the number of retransmits per packet sent. The device will mostly use power in these periods, so if the device is active for longer periods in either receive or transmit mode the expected lifetime is degraded. Most of the time the active percentage will be low due to usage of \acrfull{rai} and \acrshort{psm} mode, however if the transmit frequency is low the device will more frequently be in some kind of active state, hence the transmit and receive time will increase.

\subsubsection{\acrshort{ecl}}
The signal power level is also closely related to the power usage of the device and an important factor to monitor. You can clearly see that the power usage increases when the device enters \acrshort{ecl} 1 and 2 in the testing section \vref{section:testing}. The output of \acrshort{ecl} is taken directly from the output of \textbf{AT+NUESTATS}.

\subsubsection{Transmit power} \label{paragraph:txpower}
The amount of power used for transmit is decided by software and is key to power usage. The output retrieved from \textbf{AT+NUESTATS} show the latest transmit power level of the device. Sometimes the device will send a \acrfull{nack}, which is always sent with ${+23 \acrshort{dbm}}$ signal strength and is what \textbf{AT+NUESTATS} sometimes picks up as the latest transmit power level\cite{email:nack}. In the long-term tests with graphs from the web app the logging of transmit power might not be correct related to what transmit power level the device actually used for transmitting the packet. However, in the section on detailed tests \vref{section:detailedtest}, I will log the status of the device up to ten times per second which will give a better understanding of what is happening with the transmit power level. In addition the actual power usage will be monitored with a multimeter, giving more accurate test results.

\chapter{Testing} \label{section:testing}
I have tried to include as many tests as possible and this chapter discusses and analyses the results. The test phase began early January and was completed late March. The reason for the late test phase was due to a number of things. First of all, the hardware and software related to \acrshort{nb-iot} were not available on a stable platform until late 2017, and Telenor and Telia had limited \acrshort{nb-iot} enabled base stations. Even at the beginning of 2018, the devices used were not production ready, but at that stage the error rate was very low. Because of the early adaption I encountered some interesting results, giving an indication of the state of the technology. Given the hypothesis I needed to test the chip and the networks with good equipment and at several locations. Chapter \vref{chapter:wherewhynhow}, gave you an overview of the testing premises and the devices used.

\section{Short-term tests} \label{section:detailedtest}
I will present how the device handles normal usage and edge cases where non-default behavior is activated. I define normal operation when \acrshort{ecl} is 0 and signal power is better than -100dBm. When the \acrshort{ue} has bad reception it may retransmit packets and operation like this over longer periods is not sustainable. With normal operation it is also expected that the \acrshort{ue} transmit packets with \acrshort{rai} set and \acrshort{psm} activated.

Short-term tests is defined as a logging sequence over a short period with frequent data points. The duration of the tests are under two minutes and the sample rate is high, over 300 samples per second for some tests. These tests will show you how the device performs on a detailed level and will give a good understanding of the transmit process. The reason for including these tests are related to the actual power usage over a short period. By defining power usage statistics for different kinds of transmits it is possible to normalize the data and give an estimation of expected lifetime based on the tests. The detailed tests also will show if the network providers follow the specifications. It is very interesting to follow \acrshort{psm}, connection status, signal power and power usage over a short period and see how the device handles different scenarios.

The results are sectioned into categories and I will refer to a subset of the results in each section with a reference to a more detailed version in each caption. In this way you will be presented with one problem at a time which will give you more in-depth knowledge.

\subsection{General} \label{ssection:generaltest}
Most of the short-term tests were performed with an emphasis on power usage, rather than coverage and latency. It is however clear that high power usage is a side effect of bad coverage. All tests show that better reception gives lower latency and power usage. In one of the test cases with Telia at Q-Free I saw that the device was struggling even with good reception. Figure \vref{figure:2x150_QFREE_TELIA_SHORT}, shows that the device is not entering \acrshort{psm} mode even though the expired 120 seconds has elapsed. The device does not leave \acrshort{rrc} connected mode, hence the power consumption stays at ${6 \acrshort{ma}}$. The signal power of the device at this location was good and the \acrshort{snr} values were much like Telenor's at the same location. This test did not use \acrshort{rai}, so the expected behavior is transmit at the beginning of the program, following a period of 20 seconds in \acrshort{rrc} connected mode. After this the device should enter \acrshort{edrx} mode, only receiving data on a set interval, normally 2-25 seconds, with a number of receive transactions within each \acrshort{edrx} period. After this period the device should enter \acrshort{psm} mode if it is enabled by software.
However, in this test it looks like the device is performing \acrshort{drx} and polling for data from the network very frequent. Looking at the network connection graph you see that the device does not leave this state until it suddenly indicates loss of connection and rapidly reestablishing connection and transmits the last packet. Directly before the last transmit there is a short period of frequent receive transactions. This might be related to the loss of connection, presuming that the \acrshort{ecl} value is delayed until connection is restored. An alternative explanation is that due to the bad configuration the time of the paging window negotiation is longer than expected, resulting in a period of constantly trying to receive data from the network. Continue looking at the time of the last transmit, something interesting is happening with the transmit power. After renegotiating the connection towards the network the transmit power is adjusted from around ${+6 \acrshort{dbm}}$ to ${+23 \acrshort{dbm}}$ even though the signal power is kept at the same level. At this location most of the tests showed some kind of flaw in the network, either related to packets being dropped, long transmit times or trouble following the normal transmit procedure. It would be interesting to investigate what happened of the network side of this setup. The network becomes a black box and when the network performs out of the ordinary it is hard to debug.

\begin{figure}[H]
  \centering
  \includegraphics[width=\textwidth, height=0.4\textheight]{/Users/henninghakonsen/Dropbox/Masteroppgave/thesis/results/included_in_thesis/short_Q-Free_TELIA_5_02_precision_2018-03-13_1_0x0_150_2_100.jpeg}
  \caption[Short-term test - unusual behavior, Telia]{The figure displays one test from Telia which behaved out of the normal. See figure \vref{figure:2x150_QFREE_TELIA}, or visit \href{http://158.39.77.97:9000/\#/results/Q-Free\_TELIA\_5.02\_precision\_2018-03-13\_1\_0x0\_150\_2\_100}{webapp} \cite{online:result0}, for more details.}
  \label{figure:2x150_QFREE_TELIA_SHORT}
\end{figure}

With this in mind I will present a proper transmit procedure from Telia and Telenor. Looking at figure \vref{figure:2x150_UIO_TELIA_SHORT} and \vref{figure:2x150_QFREE_TELENOR_SHORT}, the process is totally different and is looking more like a graph representation of the specifications. Looking first at the graph from UiO with Telia there is two transmit periods of around 20 seconds, following a period of \acrshort{edrx} and then \acrshort{psm}. \acrshort{rai} was not used in this test either, hence the \acrshort{rrc} period of 20 seconds. Notice that the device is not pulling for data all the time within the \acrshort{rrc} period. It usually receives a lot of data when initiating the uplink transmit, but after the transmit it only pulls for data around 20\% of the time, hence reducing the power usage. When asking Telia about this they state that this is most likely related to \acrshort{drx} process which kicks in because there is not any data to receive\cite{email:telia}. However, I did not manage to figure out why the device performed this way when connected to Telia and not Telenor.
It is also worth noting that when the device is not in \acrshort{rrc} connected mode the statistics from \textbf{NUESTATS} become stale. This is an indication of deep sleep mode, and in \acrshort{edrx} the device should sleep between each \acrshort{edrx} period. After two \acrshort{edrx} periods the device enters \acrshort{psm} with permission from the network.

Moving focus over to Telenor there are some distinct differences from Telia's transmit process. The first thing to notice is that Telenor is pulling for data close to 100\% of the time spent in \acrshort{rrc} connected mode. This results in an increase of 5 in terms of pure power usage compared to Telia's solution. The current at this stage is comparable at around ${60-70 \acrshort{ma}}$, a little higher than Ublox specification on page 16\cite{datasheet:ubloxchip}. If you look at the graphs from the website you are able to zoom in on these periods, revealing that Telia's \acrshort{rrc} period is mostly using ${6-7 \acrshort{ma}}$, only flickering up to ${60-70 \acrshort{ma}}$ when actually receiving. Telenor's \acrshort{rrc} period however is mostly at ${50-60 \acrshort{ma}}$ and it is likely that there are different configurations in the networks resulting in actual receive time.
A part from this, the graphs have the same characteristics. One thing worth noticing is the adjusted transmit power because of the excellent reception at Q-Free. The device has ${-55 \acrshort{dbm}}$ signal power resulting in ${-24 \acrshort{dbm}}$ transmit power, almost minimum. This is clearly reflected in the power usage graph where Telenor only uses at most ${70 \acrshort{ma}}$ when transmitting, while Telia at ${+9 \acrshort{dbm}}$ transmit power is reaching ${110 \acrshort{ma}}$ at transmit time. Notice that this is close to what Ublox state in their specifications. It is however lower than calculated and this will be discussed in section \vref{section:deviations}, about deviations.

\begin{figure}[H]
  \centering
  \includegraphics[width=\textwidth, height=0.4\textheight]{/Users/henninghakonsen/Dropbox/Masteroppgave/thesis/results/included_in_thesis/short_UiO_TELIA_5_02_precision_2018-03-13_1_0x0_150_2_100.jpeg}
  \caption[Short-term test - normal behavior, Telia]{The figure displays normal behavior with two transmits over 5 minutes at UiO with Telia. See figure \vref{figure:2x150_UIO_TELIA}, or visit \href{http://158.39.77.97:9000/\#/results/Q-Free\_TELENOR\_2018-02-28\_1\_0x0\_150\_2\_100}{webapp} \cite{online:result1}, for more details.}
  \label{figure:2x150_UIO_TELIA_SHORT}
\end{figure}

\begin{figure}[H]
  \centering
  \includegraphics[width=\textwidth, height=0.4\textheight]{/Users/henninghakonsen/Dropbox/Masteroppgave/thesis/results/included_in_thesis/short_Q-Free_TELENOR_2018-02-28_1_0x0_150_2_100.jpeg}
  \caption[Short-term test - normal behavior, Telenor]{The figure displays normal behavior with two transmits over 5 minutes at Q-Free with Telenor. See figure \vref{figure:2x150_QFREE_TELENOR}, or visit \href{http://158.39.77.97:9000/\#/results/Q-Free\_TELENOR\_5.02\_2018-03-07\_0\_0x0\_40\_1\_200}{webapp} \cite{online:result2}, for more details.}
  \label{figure:2x150_QFREE_TELENOR_SHORT}
\end{figure}

\subsection{Transmit power spike}
Disregarding signal power level, one thing which was revealed after performing the tests was that there was often a spike of power usage up to ${230 \acrshort{ma}}$. Not all graphs are included in this paper, but visit \url{henninghaakonsen.me} and view results tagged with 60x1 for Telenor for example. You will see this spike in many of the graphs and it is an interesting finding thanks to the precision multimeter. The context of the occurrences is when the device is in \acrshort{rrc} connected mode, hence in active communication with the network - either \acrshort{ul} or \acrshort{dl}. As explained in paragraph \vref{paragraph:txpower}, a \acrshort{nack} is sent with ${+23 \acrshort{dbm}}$ transmit power, hence it is likely that this is what is happening at these spikes. This behavior is not mentioned in the specification, and at the moment of the tests there were more instances of this behavior than without. This behavior was also present in Telia's network, so it is possible that this is normal procedure in a transmit process. It is however somewhat strange that this happens with transmits when the signal power is excellent, since one would presume that this practice would only be in case of bad reception and signaling faults.

\subsection{Loss of connection prior to transmit} \label{ssection:ecl_255}
Some tests was performed in Telenor's network at Q-Free with interval at 40 seconds with and without \acrshort{rai}. At many of these transmits I noticed that the device indicated network disconnection directly before transmitting the packet. This resulted in a period with \acrshort{ecl} at 255 and higher power usage for a short period where the device reconnected to the network. The two figures \vref{figure:1x40_QFREE_TELENOR_1LOG_SHORT} and \vref{figure:1x40_QFREE_TELENOR_0LOG_SHORT}, represents this behavior. In these tests, there is a period of reconnection to the network, followed by the actual transmit process. This pre-procedure of reconnecting to the network operates at a current of ${45-55 \acrshort{ma}}$ and is comparable to a \acrshort{rrc} period or transmitting with very good reception.
This behavior is regardless of the \acrshort{rai} flag, hence the power usage is a lot higher compared to the actual transmit process when using the \acrshort{rai} flag. This is a root problem users and developers would have a hard time finding, and could cause battery lifetime issues. Using \textbf{power\_calculator.py} I have calculated the power usage over the pre transmit period, see result in equation \vref{equation:priortransmit}. This is actually ${0.1764 / 0.034 = ~5.2}$ times higher than a normal transmit(see \href{http://158.39.77.97:9000/\#/results/Q-Free_TELENOR_SHORT_TEST_2018-02-28_0_0x2_5_1_100}{webapp} \cite{online:result11}) with \acrshort{rai} set and excellent signal power. For each occurence of this extra procedure, it shortens the battery lifetime with approximately 5 hours, given that the application transmits every hour.

\begin{equation} \label{equation:priortransmit}
\begin{aligned}
From: 2018-03-07\quad 13:36:14.415420 \\
To: 2018-03-07\quad 13:36:19.022218 \\
((41.793363329583805\acrshort{ma} * 3.3V / 60 / 60 ) * 4.606798S) = 0.176489\acrshort{mwh} \\
\end{aligned}
\end{equation}

\begin{figure}[H]
  \centering
  \includegraphics[width=0.9\textwidth, height=0.4\textheight]{/Users/henninghakonsen/Dropbox/Masteroppgave/thesis/results/included_in_thesis/short_Q-Free_TELENOR_5_02_2018-03-07_1_0x0_40_1_200.jpeg}
  \caption[Short-term test - loss of connection, with device logging]{The figure displays one transmit over 40 seconds at Q-Free with Telenor, with device logging. See figure \vref{figure:1x40_QFREE_TELENOR_1LOG}, or visit \href{http://158.39.77.97:9000/\#/results/Q-Free\_TELENOR\_5.02\_2018-03-07\_1\_0x0\_40\_1\_200}{webapp} \cite{online:result4}, for more details.}
  \label{figure:1x40_QFREE_TELENOR_1LOG_SHORT}
\end{figure}

\begin{figure}[H]
  \centering
  \includegraphics[width=0.9\textwidth, height=0.4\textheight]{/Users/henninghakonsen/Dropbox/Masteroppgave/thesis/results/included_in_thesis/short_Q-Free_TELENOR_5_02_2018-03-07_0_0x0_40_1_200.jpeg}
  \caption[Short-term test - loss of connection, without device logging]{The figure displays one transmit over 40 seconds at Q-Free with Telenor, without device logging. Visit \href{http://158.39.77.97:9000/\#/results/Q-Free_TELENOR\_5.02\_2018-03-07\_0\_0x0\_40\_1\_200}{webapp} \cite{online:result3}, for more details.}
  \label{figure:1x40_QFREE_TELENOR_0LOG_SHORT}
\end{figure}

Because of this bug, there were some problems related to the long-term tests so further investigation was required. The device would indicate network disconnection and reconnection, resulting in \acrshort{ecl} being 255 and receive/transmit counters were reset. Looking into the issue by monitoring the transmit process towards the server it became clear that most of the time the device did not try to reconnect to the network. The behavior is not expected and is most likely related to the hardware and software of the development kit. The conclusion is that the bug does introduce reconnection some times, but it is \acrshort{ue} related and not network related. This is also based on what was described in section \vref{section:market}, that the chips produced at the point of the tests were not production ready, but good enough to test the networks.

\subsection{Signal power and \acrshort{ecl}} \label{ssection:ecltest}
Depending on the devices signal power the \acrshort{ecl} value should adjust accordingly. As discussed in section \vref{ssection:ecl}, the network provider configures thresholds for \acrshort{ecl} and the results have mostly followed these thresholds as expected. The two figures \vref{figure:1x60_UIO_TELIA_ECL_0_SHORT} and \vref{figure:1x60_UIO_TELIA_ECL_2_SHORT}, shows results from one transmit of 50 bytes over a period of 60 seconds with \acrshort{rai} set and describes the behavior of different \acrshort{ecl} values.

You should focus on the two tops at the beginning of the figures\footnote{Please visit the web app (link is supplied in the figures) to zoom into the specific periods}. The first figure indicates good signal power, but at transmit time the transmit power is bumped up from ${+9 \acrshort{dbm}}$ to ${+23 \acrshort{dbm}}$, maybe because of a \acrshort{nack}. The transmit graph shows that the transmit process is relatively short, only covering approximately 1.5 seconds.
In addition the device was actually only active for 15-20\% of that time, resulting in actual transmit time of \textasciitilde0.3 seconds. The following equation \vref{equation:ecl0}, shows the power usage of the transmit with \acrshort{ecl} 0 using the program \textbf{power\_calculator.py}.

\begin{equation} \label{equation:ecl0}
\begin{aligned}
From: 2018-03-16 12:35:55.90 \\
To: 2018-03-16 12:35:57.54 \\
((24.77\acrshort{ma} * 3.3V / 60 / 60 ) * 1.64s) = ~0.037\acrshort{mwh} \\
\end{aligned}
\end{equation}

This next figure is showing a transmit where the signal power is poor, hence the \acrshort{ecl} is 2. Not surprisingly the transmit power is at ${+23 \acrshort{dbm}}$ and there is a clear difference compared to the previous result. First of all the transmit process endures longer, around 5 seconds, hence the device is retransmitting the packet to ensure that it will be received at the \acrshort{enb}. The following equation \vref{equation:ecl2}, shows the power usage of the transmit with \acrshort{ecl} 2.

\begin{equation} \label{equation:ecl2}
\begin{aligned}
From: 2018-03-16 13:13:42.28 \\
To: 2018-03-16 13:13:47.91 \\
((80.84\acrshort{ma} * 3.3V / 60 / 60 ) * 5.63s) = ~0.417\acrshort{mwh} \\
\end{aligned}
\end{equation}

By comparing the two transmit operations we see that the second transmit uses \textasciitilde{}11 times as much power as the first. In my experience the first transmit is the most accurate towards what actually is going on most of the time, but if the \acrshort{ue} has bad signal power the battery lifetime will be heavily affected. In section \vref{ssection:transmitprocedure}, I described the theoretical disadvantage of different \acrshort{ecl} values, however I think that this comparison is more accurate towards normal behavior. The results from Telia was included for this observation since the results from the two networks were very similar.

\begin{figure}[H]
  \centering
  \includegraphics[width=0.9\textwidth, height=0.4\textheight]{/Users/henninghakonsen/Dropbox/Masteroppgave/thesis/results/included_in_thesis/short_UiO_TELIA_5_02_precision_2018-03-16_1_0x2_60_1_50.jpeg}
  \caption[Short-term test - \acrshort{ecl} 0]{The figure displays one transmit over 60 seconds at UiO with Telia, with \acrshort{ecl} 0. See figure \vref{figure:1x60_UIO_TELIA_ECL_0}, or visit \href{http://158.39.77.97:9000/\#/results/UiO\_TELIA\_5.02\_precision\_2018-03-16\_1\_0x2\_60\_1\_50}{webapp} \cite{online:result5}, for more details.}
  \label{figure:1x60_UIO_TELIA_ECL_0_SHORT}
\end{figure}

\begin{figure}[H]
  \centering
  \includegraphics[width=0.9\textwidth, height=0.4\textheight]{/Users/henninghakonsen/Dropbox/Masteroppgave/thesis/results/included_in_thesis/short_UiO_TELIA_5_02_precision_+30dBm_att_2018-03-16_1_0x2_60_1_50.jpeg}
  \caption[Short-term test - \acrshort{ecl} 2]{The figure displays one transmit over 60 seconds at UiO with Telia, with \acrshort{ecl} 2. See figure \vref{figure:1x60_UIO_TELIA_ECL_2}, or visit \href{http://158.39.77.97:9000/\#/results/UiO\_TELIA\_5.02\_precision\_+30dBm\_att\_2018-03-16\_1\_0x2\_60\_1\_50}{webapp} \cite{online:result6}, for more details.}
  \label{figure:1x60_UIO_TELIA_ECL_2_SHORT}
\end{figure}

\subsection{Transmit comparison}
In an attempt to compare the two networks I did many similar tests with both networks. All transmits used the same setup and the signal power was altered using attenuators. One thing to note is that since the tests were performed at Q-Free where the signal power of Telenor's network is exceptional, the device never entered \acrshort{ecl} 1 or 2 with Telenor's network. There was however an interesting result and we will begin looking at figure \vref{figure:transmit_comparison_0}, from the transmits without \acrshort{rai} set.

There are a few things to notice in this chart. First of all there are two groupings, one with Telenor's transmits and another with Telia's. As we have seen before Telia's transmits use less time in \acrshort{rrc} connected mode, hence lowering the power usage. Telia uses around 50\% less power at the first two transmits, regardless of byte size. It is not until the last transmit where Telia's transmits uses more power, and this is because the device has entered \acrshort{ecl} 2 and retransmits occurs.
The next section will focus on packet size, but there are indications that packet sizes does influence the power usage to some degree.

Moving over to figure \vref{figure:transmit_comparison_1}, with transmits with \acrshort{rai} set, the results are more similar between the two network providers when the device is in \acrshort{ecl} 0. By looking closely at the first transmits from Telia there is a bigger gap between packet sizes, and this is mainly because the \acrshort{rrc} period is not influencing the results, thus gaining more information about the actual transmit process. You can view the actual transmit graphs at \url{henninghaakonsen.me} where they are tagged with \textbf{60\_1}.

\begin{figure}[H]
  \centering\includegraphics[width=\textwidth,height=8cm]{/Users/henninghakonsen/Dropbox/Masteroppgave/thesis/latex/images/transmitcomparison_0}
  \caption[Short-term test - comparison without \acrshort{rai}]{The figure compares transmits without \acrshort{rai}. The y-axis shows how much power was used, in \acrshort{mwh}.}
  \label{figure:transmit_comparison_0}
\end{figure}

\begin{figure}[H]
  \centering\includegraphics[width=\textwidth,height=8cm]{/Users/henninghakonsen/Dropbox/Masteroppgave/thesis/latex/images/transmitcomparison_1}
  \caption[Short-term test - comparison with \acrshort{rai}]{The figure compares transmits with \acrshort{rai}. The y-axis shows how much power was used, in \acrshort{mwh}.}
  \label{figure:transmit_comparison_1}
\end{figure}

\subsection{Packet size} \label{ssection:packetsize}
The typical packet size for an \acrshort{iot} device is often small, typically under 100 bytes, but for some cases bigger packets might be required and I have performed a specific test to see what the packet size does with a transmit. The test is performed with the program \textbf{nbiot\_labtest\_details.py} and the figures, \vref{figure:details_0} and \vref{figure:details_1}, displays two executions of the program - one with \acrshort{rai} and one without. It was important to test both transmit modes to see how much the transmit process with different packet sizes costs in terms of power usage.

Looking at the first figure, transmits without \acrshort{rai}, you see that the whole transmit process without \acrshort{rai} endures approximately 20-30 seconds, while the actual transmit process only takes 0.5-2 seconds depending on the signal power level. Investigation all aggregated lines it is clear that the bigger part of the power consumption comes from overhead of the \acrshort{rrc} process. There is a slight increase in power usage while increasing the packet size, but if your application transmits packets without \acrshort{rai} flag the packet size is not crucial to the power usage.

Moving over to the next graph there is a noticeable difference. Not surprisingly there is a general decrease in power usage when using \acrshort{rai}. The average aggregated value of packets transmitted with 25 bytes, shows a decrease of 240\%. This results in less overhead related to the transmit, hence the power usage from the transmit has bigger influence on the graphs. Looking at the average line in pink there is a steady increase in power usage related to the packet size. Comparing 25 and 512 bytes, there is a power increase of 217\% which sounds logical since a bigger payload requires several data packets. It is interesting to look at the aggregated sum of each byte size category. According to \cite{online:rohde}, the \acrfull{mps} is 680 bits, which is 85 bytes. Looking at the aggregated sum line two spikes appear which is likely related to the \acrshort{mps}. There is a spike at 100 bytes, which is directly after the \acrshort{mps}. There is another spike from 400 to 512 bytes, which may be related to the \acrshort{mps} of the chip which is 512 bytes.

\begin{figure}[H]
  \centering
  \includegraphics[width=0.9\textwidth, height=0.4\textheight]{/Users/henninghakonsen/Dropbox/Masteroppgave/thesis/latex/images/details_0.png}
  \caption[Short-term test - packet size comparison without \acrshort{rai}]{The figure compares different packet sizes without \acrshort{rai} set. Visit \href{http://158.39.77.97:9000/\#/results/UiO\_TELIA\_long\_term\_2018-03-22\_0x0\_30\_20}{webapp} \cite{online:result7}, for more details.}
  \label{figure:details_0}
\end{figure}

\begin{figure}[H]
  \centering
  \includegraphics[width=0.9\textwidth, height=0.4\textheight]{/Users/henninghakonsen/Dropbox/Masteroppgave/thesis/latex/images/details_1.png}
  \caption[Short-term test - packet size comparison with \acrshort{rai}]{The figure compares different packet sizes with \acrshort{rai} set. Visit \href{http://158.39.77.97:9000/\#/results/UiO\_TELIA\_long\_term\_2018-03-22\_0x2\_30\_20}{webapp} \cite{online:result8}, for more details.}
  \label{figure:details_1}
\end{figure}

\subsection{Sensor reboot} \label{ssection:reboottest}
There are a number of reasons why the device would perform a reboot, especially if the application is operating over several years. One reason for a reboot is because of some software or hardware fault, which could occur even though it should not happen. Another reason is that the developer reboots the device if it has stalled or some logic kicks in. The main reason to investigate what happens in a reboot is to see how long to network connection takes. If this process is long and cumbersome it will affect the power usage, hence the battery lifetime decreases. I will elaborate why and when your application should consider rebooting the device in section \vref{section:guidelines}.

I performed several reboot tests with both network providers and different signal power levels. I have included two results, one from Telenor and one from Telia. Since the device reboots and the program looses the connection to the chip it is not possible to log the status of the device with \textbf{NUESTATS} so you will only be able to see the power usage in this test.

Figure \vref{figure:telenor_reboot}, presents the first example which is in Telenor's network without any attenuators. The graph displays 120 seconds which includes the whole process of rebooting and connecting to the network. The first thing to notice is the long period of what looks like \acrshort{rrc} connected mode. Following is what looks like the actual connection process about half way into the graph and ending with a \acrshort{rrc} period until the timer runs out and the device enters \acrshort{psm} mode. The whole process uses ${~5.1 \acrshort{mwh}}$, which is the same as around 28 transmits with 100 bytes payload with \acrshort{rai} set using the results from the test of packet size. Assuming one transmit every hour, this reboot process uses approximately the same amount of power as a days worth of uptime.

Moving over to Telia's result without any attenuators, see figure \vref{figure:telia_reboot}, you might notice that this result only covers 60 seconds. This is because the reboot process using Telia's network was much shorter and since the reboot process is the focus, the log duration is kept to the minimum. Directly after reboot the device performs a set of transmits, followed by a period with \acrshort{rrc} connected mode. After this process the device enters \acrshort{psm} and the device has connected to the network. The whole process uses ${0.55 \acrshort{mwh}}$, which is around ten times less than with Telenor's network. It is interesting to see the differences the network provider introduces to the network connection process, since this is mainly due to network configurations. One explanation of the lower power consumption is likely related to the power consumption during the \acrshort{rrc} period. Section \vref{ssection:generaltest}, showed that Telia uses around 60\% less time in \acrshort{rrc} connected mode, hence reducing the power consumption drastically. In addition, the duration of the connection process is twice as long with Telenor. The results are consistent throughout all reboot tests and gives an indication that Telia has used the configuration to reduce the power consumption on reboot and in these \acrshort{rrc} periods.

\begin{figure}[H]
  \centering
  \includegraphics[width=0.9\textwidth, height=0.4\textheight]{/Users/henninghakonsen/Dropbox/Masteroppgave/thesis/results/included_in_thesis/short_UiO_TELENOR_5_02_precision_reboot_2018-03-16_0_0x0_120_1_0.jpeg}
  \caption[Short-term test - device reboot, Telenor]{The figure shows a reboot of the development kit with Telenor. Visit \href{http://158.39.77.97:9000/\#/results/UiO\_TELENOR\_5.02\_precision\_reboot\_2018-03-16\_0\_0x0\_120\_1\_0}{webapp} \cite{online:result9}, for more details.}
  \label{figure:telenor_reboot}
\end{figure}

\begin{figure}[H]
  \centering
  \includegraphics[width=0.9\textwidth, height=0.4\textheight]{/Users/henninghakonsen/Dropbox/Masteroppgave/thesis/results/included_in_thesis/short_UiO_TELIA_5_02_precision_reboot_2018-03-16_0_0x0_60_1_0.jpeg}
  \caption[Short-term test - device reboot, Telia]{The figure shows a reboot of the development kit with Telia. Visit \href{http://158.39.77.97:9000/\#/results/UiO\_TELIA\_5.02\_precision\_reboot\_2018-03-16\_0\_0x0\_60\_1\_0}{webapp} \cite{online:result10}, for more details.}
  \label{figure:telia_reboot}
\end{figure}

\subsection{Downtime test} \label{ssection:downtimetest}
If the network is down because of a power outage or similar breakdowns the \acrshort{ue} might be disconnected from the network if there are no other \acrshort{enb}'s in the close proximity. There was a period in the long-term tests where the device actually lost connection to the network, even though the signal strength showed otherwise. In figure \vref{figure:downtime}, you can see a longer monitoring sequence. At first the device transmits data, but after a short while it is disconnected and the subsequent results gives an indication of what happens if the device looses connection. The receive/transmit timers does not change, hence it looks like the device does not use any effort trying to reconnect to the network. I will discuss different approaches to this problem in section \vref{section:guidelines}.

\begin{figure}[H]
  \centering
  \includegraphics[width=1\textwidth, height=0.6\textheight]{/Users/henninghakonsen/Dropbox/Masteroppgave/thesis/latex/images/downtime.pdf}
  \caption[Short-term test - downtime]{Example of disconnection from the network}
  \label{figure:downtime}
\end{figure}

\cleardoublepage
\section{Short-term figures}
\subsection{General} \label{ssection:general}
\begin{figure}[H]
  \centering
  \includegraphics[width=0.9\textwidth, height=0.8\textheight]{/Users/henninghakonsen/Dropbox/Masteroppgave/thesis/results/included_in_thesis/full_Q-Free_TELIA_5_02_precision_2018-03-13_1_0x0_150_2_100.jpeg}
  \caption{Short-term figure - unusual behavior, Telia}
  \label{figure:2x150_QFREE_TELIA}
\end{figure}

\begin{figure}[H]
  \centering
  \includegraphics[width=0.9\textwidth, height=0.8\textheight]{/Users/henninghakonsen/Dropbox/Masteroppgave/thesis/results/included_in_thesis/full_UiO_TELIA_5_02_precision_2018-03-13_1_0x0_150_2_100.jpeg}
  \caption{Short-term figure - normal behavior, Telia}
  \label{figure:2x150_UIO_TELIA}
\end{figure}

\begin{figure}[H]
  \centering
  \includegraphics[width=0.9\textwidth, height=0.8\textheight]{/Users/henninghakonsen/Dropbox/Masteroppgave/thesis/results/included_in_thesis/full_Q-Free_TELENOR_2018-02-28_1_0x0_150_2_100.jpeg}
  \caption{Short-term figure - normal behavior, Telenor}
  \label{figure:2x150_QFREE_TELENOR}
\end{figure}

\subsection{Downtime prior to transmit} \label{ssection:downtimeprior}
\begin{figure}[H]
  \centering
  \includegraphics[width=0.9\textwidth, height=0.8\textheight]{/Users/henninghakonsen/Dropbox/Masteroppgave/thesis/results/included_in_thesis/short_Q-Free_TELENOR_5_02_2018-03-07_0_0x0_40_1_200.jpeg}
  \caption{Short-term figure - loss of connection, without device logging}
  \label{figure:1x40_QFREE_TELENOR_0LOG}
\end{figure}

\begin{figure}[H]
  \centering
  \includegraphics[width=0.9\textwidth, height=0.8\textheight]{/Users/henninghakonsen/Dropbox/Masteroppgave/thesis/results/included_in_thesis/full_Q-Free_TELENOR_5_02_2018-03-07_1_0x0_40_1_200.jpeg}
  \caption{Short-term figure - loss of connection, with device logging}
  \label{figure:1x40_QFREE_TELENOR_1LOG}
\end{figure}

\subsection{\acrshort{ecl}} \label{ssection:ecllevel}
\begin{figure}[H]
  \centering
  \includegraphics[width=0.9\textwidth, height=0.8\textheight]{/Users/henninghakonsen/Dropbox/Masteroppgave/thesis/results/included_in_thesis/full_UiO_TELIA_5_02_precision_2018-03-16_1_0x2_60_1_50.jpeg}
  \caption{Short-term figure - \acrshort{ecl} 0}
  \label{figure:1x60_UIO_TELIA_ECL_0}
\end{figure}

\begin{figure}[H]
  \centering
  \includegraphics[width=0.9\textwidth, height=0.8\textheight]{/Users/henninghakonsen/Dropbox/Masteroppgave/thesis/results/included_in_thesis/full_UiO_TELIA_5_02_precision_+30dBm_att_2018-03-16_1_0x2_60_1_50.jpeg}
  \caption{Short-term figure - \acrshort{ecl} 2}
  \label{figure:1x60_UIO_TELIA_ECL_2}
\end{figure}

\section{Long-term tests} \label{section:longtermtest}
A device operating for a long time is prone to many bugs and software quality is essential. This section gives an overview of how the network handles continuous transmits over a longer period. I have tested the device at all locations with both network providers where the device managed to connect to the network. I define long-term tests as a set of operations over a long period, typically one or several days. When trying to test many scenarios, the test period per network provider and location was limited. However, the results are comparable to the short-term tests and there are clear indications of the networks pros and cons. The reason for the long-term tests is to monitor the behavior of the chip over a longer period, which gives statistics at a more abstract level, and can be combined with the short-term tests. With this relation I will give you an overview of the most common states of a device and how different aspects reflect the behavior. I had some problems logging correct behavior of the device. It seems that at random points in time it looks as if the device looses connection to the network resulting in \acrshort{ecl} set to 255, and receive/transmit timers are reset. It is at this stage unclear what evokes this behavior, but one reason might be related to \acrshort{psm} periods as the device shuts down all internal processes only keeping the clock and some logic alive. There were some progress to this problem in late March when I used the program \textbf{nbiot\_labtest.py}, for the long-term tests. This program constantly pulls the chip for statistical data, hence it is reasonable to assume that the device is more active. Even with this program there were signs of the same behavior and because of this abnormality some of the results are harder to make sense of, especially receive/transmit time. However, the results were very much like the results from the short-term tests and I will use the good results to describe how the network performs. Most of the figures included in this section is taken from the web application so if you are unfamiliar with the format please refer to section \vref{sssection:layout}, for a more detailed description of the graphs. For the best experience I advice you to use the web application to view the data, but the most interesting findings will be included in each apprioriate section.

\subsection{Signal power and latency}
One of the key features of \acrshort{nb-iot} is increased link budget. Devices should have good reception in normally bad coverage areas, such as tight city areas and far away mountain sides. The transmit power is ${+23 \acrshort{dbm}}$ which is very good, considering that a theoretical ${+3 \acrshort{dbm}}$ step is actually doubling in power output. The signal power is not only related to the distance between the \acrshort{ue} and the base station, but also the obstacles between. The map \vref{figure:map_qfree}, shows that the distance between the \acrshort{ue} and the \acrshort{enb}'s is low, but the signal power was quite different. With Telenor, the reception was normally at ${-60 \acrshort{dbm}}$, while with Telia, the reception was normally at ${-80 \acrshort{dbm}}$.
The main reason for the big difference is due to what lies between the sender and receiver. There are buildings and obsticles between Q-Free and Telia's \acrshort{enb}, but there is very few obsticles between Q-Free and Telenor's \acrshort{enb}. It is reasonable to assume that the two network providers use the same technology to offer \acrshort{nb-iot}, but it is probably not from the same producer which might explain the big difference in reception as well.

The following two sections gives you insight in two longer transmit sequences. Because of the issues related with the long-term test I have mostly focused on one of the sequences with Telenor since the tests revealed that their network is the most stable of the two.

\subsubsection{Telenor}
This section will present a transmit session from UiO over 5 days, 23.03.18-28.03.18. The figures \vref{figure:uio_telenor_lc} and \vref{figure:uio_telenor_stat}, is taken from the web application and will be used to point out certain aspects of the sequence. The first section displays two steady lines with signal power around ${-95 \acrshort{dbm}}$ and latency between 1.5-3.5 seconds on average. This classifies as quite good signal and resembles normal behavior. A thing I noticed is the flow of the latency line. There is a clear interval between the high latency spikes and the low. I have not been able to find the origin of this behavior, but it has most likely something to do with the process in the core network since both the server and the client in this setup behaves the same way at all times. Another indication that it originates from the core network is that my supervisor, Q-Free's R\&D manager, Ola Martin, also has seen this behavior in his tests\cite{person:ola}. Another interesting thing to view is receive/transmit time in the second figure. The transmit graph shows that a normal transmit uses approximately 0.3 seconds which is relatable to the short-term tests. Also looking at the receive time graph the normal receive time between two transmits are around 21 seconds since these transmits were not using the \acrshort{rai} flag. This is also very well in terms of what was presented in the short-term tests. This is useful information since it is now clear that this is not only a one time occurrence, but the normal behavior of a receive/transmit process without \acrshort{rai} set.

The same period shows three clear latency spikes up towards 10 seconds without any indication of bad signal power. However, at each latency spike a similar spike in receive time is present and this indicates that something happened in the network which the device needed to listen in on, hence the transmit towards the server is delayed. The specification states a maximum latency of 10 seconds, so these transmits are on the edge of what is normal behavior. After a while, around 16:00 23.03, the trend shifts. The signal power is the same, but the latency suddenly fluctuates up towards 5-6 seconds and this happens with the same interval flow as previously. To try to see if the behavior is related to the test program I restarted the program at, 20:36 25.03, now with \acrshort{rai} flag set. The same behavior continued until I restarted the development kit at 14:00 26.03, where the signal power and latency normalizes to what classifies as normal behavior. A thing you might have noticed is the drop in receive/transmit time. There is a noticeable behavior change in receive time when \acrshort{rai} is set and this is very logical since the device goes to sleep directly after the transmit process has finished. Even though the transmit process only covers \textasciitilde0.3 seconds there are still some overhead related to the transmit process so the typical receive time for Telenor is 2-3 seconds, which is approximately ten times lower than with \acrshort{rai} not set. It is hard to see from the figure, but the transmit time is also reduced to some extent while using the \acrshort{rai} flag. This is related to lowering the network communication overhead of a transmit. The average transmit time with \acrshort{rai} set is approximately 0.15-0.2 seconds, which is around 50\% less than without \acrshort{rai} set. Since the transmit process involves higher power usage a decrease in time usage is valuable.

From 10:00 to 12:00 26.03.18 three spikes in the receive time appear, which might indicate trouble with the connection towards the \acrshort{enb}. If the device detects loss of connection it will remain in \acrshort{rrc} connected mode for a longer period, hence pulling data on a regular basis which consumes very much power.

At 12:14 27.03.18 I added a ${20 \acrshort{dbm}}$ attenuator to the development kit hence the signal power was reduced to around ${-120 \acrshort{dbm}}$. The reason why I add attenuators is because it is interesting to see how the network performs in these conditions, even though it will hopefully not be the case of most applications. First of all, the base latency is around the same, but the interval pointed out earlier is not as clear as before and the overall latency is higher. More interesting is to watch the receive/transmit time graphs, as they show drastic behavior changes. The transmit time graph shows that the device retransmits packets very often. Many transmits uses around 2 seconds, while others are between 4-8 seconds. This is close to what was described in section \vref{ssection:transmitprocedure}, meaning that in addition to time spent in transmit mode the transmit it self also uses maximum power for each transmit. This behavior is extremely battery deficient and will cause poor lifetime.

Moving over to the graph with receive time indicates that the minimum receive time is still 2-3 seconds, but now the line is very unstable. There are more tendencies to spikes as the device has to use more time communicating with the network because of the poor reception. However, the receive time is not affected to the same degree as the transmit time and is kept under 10 seconds for most of the transmits. Looking at the two graphs the transmit time increased with approximately ${2.5s / 0.2s = 12.5}$, while receive time only increased with approximately ${5s / 2.5s = 2}$. This is not surprising given that the device retransmits packets frequently at \acrshort{ecl} 1 and 2 to achieve higher receive rate at the \acrshort{enb}.

\begin{figure}[H]
  \centering
  \includegraphics[width=0.9\textwidth, height=0.5\textheight]{/Users/henninghakonsen/Dropbox/Masteroppgave/thesis/latex/images/uio_telenor_lc.pdf}
  \caption[Long-term test - Telenor 23.03.18-28.03.18, signal power and latency]{The figure displays a transmit session at UiO with Telenor over 5 days, 23.03.18-28.03.18, with the signal power and latency graph. Visit \href{http://158.39.77.97:9000/\#/nodes/id2}{webapp}, for more details.}
  \label{figure:uio_telenor_lc}
\end{figure}

\begin{figure}[H]
  \centering
  \includegraphics[width=0.9\textwidth, height=0.5\textheight]{/Users/henninghakonsen/Dropbox/Masteroppgave/thesis/latex/images/uio_telenor_stat.pdf}
  \caption[Long-term test - Telenor 23.03.18-28.03.18, statistics]{The figure displays a transmit session at UiO with Telenor over 5 days, 23.03.18-28.03.18, with the statistics graphs. Visit \href{http://158.39.77.97:9000/\#/nodes/id2}{webapp}, for more details.}
  \label{figure:uio_telenor_stat}
\end{figure}

\subsubsection{Telia}
Transitioning over to Telia I will present one of the sequences recorded with Telia's network at UiO, from 20.03.18 to 23.03.18. The setup is similar and there are obvious differences. The two figures \vref{figure:uio_telia_lc} and \vref{figure:uio_telia_stat}, illustrates the results and the beginning of the first figure shows that the graphs are comparable to Telenor's result. However, as you might have noticed most of the transmits are less stable, which affects the results. Since the network does not give any feedback, it is hard to state the reason for this instability, but referring to section \vref{section:market}, we know that Telia deployes \acrshort{nb-iot} in-band within \acrshort{lte}. As previously stated in section \vref{section:nb-iot-deployment}, if there are many users in a certain area, this can cause poor performance, hence the results reflect what might be this kind of behavior.

In addition the problem with reconnection is worse with Telia's network. This is probably related to the instabilities we have seen. Since the results are degraded I want you to focus on the results from the previous section as a kind of benchmark to what is the closest to stabile behavior at this moment. However, the results from Telia shows an improvement in time spent in receive mode with \acrshort{rai} not set. Looking closer at the last part of the graph there is a steady sequence without reconnections. This show that when the transmit process behaves properly the device spends around 6 seconds in receive mode between each transmit. As stated in section \vref{ssection:generaltest}, Telia uses approximately 60\% less time in \acrshort{rrc} connected mode resulting in lowered power consumption. At first I thought that this was the reason for the instabilities, but the behavior is the same with \acrshort{rai} set. Only considering transmit and receive time, the two networks perform more or less identical when \acrshort{rai} is set.

\begin{figure}[H]
  \centering
  \includegraphics[width=0.9\textwidth, height=0.4\textheight]{/Users/henninghakonsen/Dropbox/Masteroppgave/thesis/latex/images/uio_telia_lc.pdf}
  \caption[Long-term test - Telia 20.03-23.03, signal power and latency]{The figure displays a transmit session at UiO with Telia over 3 days, 20.03-23.03, with the signal power and latency graph. Visit \href{http://158.39.77.97:9000/\#/nodes/id1}{webapp}, for more details.}
  \label{figure:uio_telia_lc}
\end{figure}

\begin{figure}[H]
  \centering
  \includegraphics[width=0.9\textwidth, height=0.4\textheight]{/Users/henninghakonsen/Dropbox/Masteroppgave/thesis/latex/images/uio_telia_stat.pdf}
  \caption[Long-term test - Telia 20.03-23.03, statistics]{The figure displays a transmit session at UiO with Telia over 3 days, 20.03-23.03, with the statistics graphs. Visit \href{http://158.39.77.97:9000/\#/nodes/id1}{webapp}, for more details.}
  \label{figure:uio_telia_stat}
\end{figure}

\subsection{Cell selection}
When the device has bad reception it will try to reselect which cell it is connected to. I wanted to test how this affects the battery lifetime since this might happen frequently if the device is located equally far from two cells with approximately the same signal power. It might then jump back and forth between these cells which could drain the battery. However, because of the limited \acrshort{nb-iot} enabled \acrshort{enb}'s I was not able to test this hypothesis. An attempt to connect to a network leads to higher power usage[\ref{ssection:reboottest}], hence reconnecting to a new \acrshort{enb} will probably use a similar amount of power.

\subsection[\acrlong{tsr}]{\acrfull{tsr}}
For an \acrshort{iot} application receive rate is not a priority, but it is important with stability and high uptime. All transmits towards the server includes a message id and this is used this to generate statistics about the \acrshort{tsr}. The program \textbf{uptime.py} takes in three parameters, node id(-c), start date(-start) and end date(-end). If you specify the start and end dates the program will only include data within the given period. This option came handy due to issues related to the downtime on the server.
The program calculates the \acrshort{tsr} for sequences with a total number of transmits higher than 5 for the given period and prints a summary of the results. I investigated all continuous sequences and used that to base the results upon. In listing \vref{code:uptimerun}, you can see the output of one of the results, which displays data for id3 between \textbf{2018-03-13T19:30:00} and \textbf{2018-03-15T09:00:00}. Here you see that there is one period with a total transmits of 438 and actual received transmits of 426, which results in a \acrshort{tsr} of 97.26\%.

\begin{lstlisting}[caption={Example of uptime.py}, label={code:uptimerun}, language=Bash]
  Statistics about collection:  id3
  Transmits:  438
  Received transmits:  426
  Uptime:  97.26%
\end{lstlisting}

I have looked at the sequences from UiO and can see that the \acrshort{tsr} for Telenor and Telia is pretty similar to the listing example. There are some packets which are lost, which surprised me. Considering the number of devices connected to \acrshort{nb-iot} at the point of the tests, the \acrshort{tsr} should have been higher. A packet loss of \textasciitilde3\% is not very high, but if you imagine devices located in cities with device density of the technical specified limit of 50 thousand, the packet loss will probably increase resulting in degraded quality of service. One does not want the developers to have to implement retransmit logic in their applications so it will be interesting to see how the \acrshort{tsr} changes as more devices connect to the network.

\chapter{Deviations} \label{section:deviations}
\section{Imprecise clock}
The AT command \textbf{AT+CCLK?} retrieves the time from the network. The timestamp is precise, but there were signs that after a while the time skewed and the latency on the server became negative. If you were to rely on the internal clock for your application, it would require time synchronization on regular intervals. Since this problem is also time dependent, meaning that after longer uptime the timestamp will be more skewed, the different \acrshort{ue}'s in your setup will have different offset to the correct time. The \acrshort{ue} used was not able to sync the clock at specific intervals, hence I do not have any recommendations to how often one would perform the synchronization.

If your application requires very precise timestamps and you need to use the internal clock, a synchronization process is required. However, if the application tollerates some slack you can use the server timestamp instead of the timestamp retrieved from the \acrshort{ue}. The server timestamp is affected by the latency in the network, but given that the \acrshort{ue} has good coverage the latency has stayed under ten seconds, which may be good enough for some applications.

\section{Imprecise \textbf{NUESTATS}}
I believe that \textbf{NUESTATS} has given great feedback about the behavior of \acrshort{nb-iot}. As stated, the command is not 100\% accurate and there were certain issues related to \acrshort{ecl} and receive/transmit time, but in the end, when I combined the results from the test applications it is clear what the device actually does. As stated in sections \vref{ssection:manufacture} and \vref{section:longtermtest}, the short-term tests gives the best picture of the state of the network at the time of the testing phase.

\section{Network load}
Since the \acrshort{nb-iot} network was not in production at the time of the tests, the network was not heavily loaded. This means that the latency could be a bit higher with more general activity on the network. When the network is available the latency should be low and that is my impression on \acrshort{nb-iot} as well.

\section{Network density}
Since the network was not in production, the density of the cell towers with \acrshort{nb-iot} were limited. In the future all cells will implement \acrshort{nb-iot}, meaning that the network will perform better. This means better coverage and lower latency given that the connected devices are spread over several \acrshort{enb}'s.

\section{Theoretical vs. practical power usage}
The theory behind conversion between \acrshort{dbm}, \acrshort{ma} and \acrshort{mwh} does not apply to the expected extent. I believe that calculating power usage based on the theory gives an indication of the power consumption, but it is only when using the appropriate tools that you will achieve the best understanding of power usage.

\chapter{Conclusions and future work}                     %% ... or Konklusjon
This chapter sums up the most important features of \acrshort{nb-iot} and gives a status report on the current situation. I will give a short introduction to best practice guidelines, as well as combining the results.

\section{Real world application guidelines} \label{section:guidelines}
The testing phase has given a great overview of what one can expect of \acrshort{nb-iot}. There are many possiblilites and with the results in mind I will give you a short introduction to what I believe is best practice guidelines for \acrshort{nb-iot} in regard to a real world application. I will use Q-Free's parking sensors specifications to estimate power usage. The sensor is equipped with two 3600\acrshort{mah} batteries at 3.6 volts, which adds up to a maximum 25 920\acrshort{mwh}.

Many developers will have a hard time deciding the transmit interval because it is difficult to predict the battery usage without doing any tests. Before calculating the power usage it is necessary to specify what packet size is recommended. Section \vref{ssection:packetsize}, showed that there was a steady power usage increasement when increasing the packet size, which is logical. By lowering the packet size you will be able to increase the transmit interval if it is necessary. I recommend that your application normally transmits small packets, around 50 bytes, and if necessary have an alternate packet, between 100-200 bytes, which the device can transmit from time to time. The alternate packet may contain a more detailed overview of the state of the device so that the server knows the state of each sensor. This is useful for maintenance and gives an indicator if something is wrong. When transmitting this larger packet you may also unset the \acrshort{rai} flag to allow for downlink communication. If your application only uses \acrshort{rai}, and \acrshort{edrx} is not enabled, it will never know if there is downlink data from the network. By utilizing the opertunity with the alternate packets your application will be able to receive downlink data which might be useful in some cases. A good example of a downlink message is for configuration purposes. Maybe you want to add data properties to your applications, or change the transmit interval. If so, you can do this if your application supports it and you disable \acrshort{rai} regularly. Keep in mind that the application can't be changed after it is installed, so you are better of with many safety mechanisms in case of errors.

Another dilemma related to an application running for several years is network downtime. There will be periods where the device will loose connection to the network, either because of a power outage or because of unexpected behavior. Your application will need to handle these situations and I belive there are a couple of approaches to this problem. If the application detects network failure it should go into a configured reconnection period. This period can for example cover one day and do a number of things. Firstly the device will try to reconnect to the network before a set of transmits. It is wise to reconnect at an exponential rate, for example reconnecting at a rate equal to the power of 2. Imagine you loose network connection at 10:05, I recommend trying to reconnect to the network a set period prior to the following transmit. If the connection is not up at the time of the transmit the application should reconnect at the second following transmit, and then the fourth following transmit. In addition your application can reboot the device after a number of reconnection attempts. Rebooting the device will be a last resort option and should not be reattempted many times since it drains the battery. If the device does not manage to reinitiate connection towards the network after a day your application can restart the reconnection period, but you might want to reduce the number of reconnection attempts to prehibit more energy waste.

Only considering transmits of 100-200 bytes and with \acrshort{rai} set, the device consumes around ${0.2 \acrshort{mwh}}$ per transmits\cite{online:result8}. There has been transmits which used ${<0.05 \acrshort{mwh}}$\cite{online:result12}, but at the time of the testing this was not consistent and I will not base the following calculation on these results. It is good to see that the device actually can perform better, especially considering that the \acrshort{ue} used was not specifically developed for \acrshort{nb-iot}.
I will continue using ${0.2 \acrshort{mwh}}$ as the base transmit consumption and based on the tests I have concluded that one transmit per hour is a good tradeoff. Given that there are similar circumstances for ten years, this results in a total power consumption of, ${0.2\acrshort{mwh} * 24 * 365 * 10 = 17 520 \acrshort{mwh}}$.

Objectivly, it is not likely that the tested setup would perform without flaws for ten years. I have create an excel spreadsheet(\textbf{longterm\_calulations.xlxs}), which gives an overview of the general results, categorized by the performance of one transmit and accumulated up to twenty years. See figure \vref{chart:longtermcalculation}, for an overview of the accumulated values for the different performance levels.

\begin{figure}[H]
  \centering
  \includegraphics[width=0.9\textwidth, height=0.4\textheight]{/Users/henninghakonsen/Dropbox/Masteroppgave/thesis/latex/images/longterm_calculation.pdf}
  \caption[Long-term accumulated performance chart]{The figure gives an overview of the accumulated values for the different performance levels.}
  \label{chart:longtermcalculation}
\end{figure}

In addition I have added a simple form for splitting the amount of transmits over these categories. The spreadsheet is located in the 'general' folder of the thesis github page and lets you select different percentages of good and bad transmits. One out of twentyfour good transmits does not use \acrshort{rai}, and for simplicity the calculation uses the amount of bad transmits as basis for the reboot attribute. The rate of reboots is 10\% of the percentage with bad transmits, which means one reboot per twentyfourth transmit, 10\% of the time. In table \vref{table:longtermcalculation}, you can see what I believe is the closest to what the network would perform today. Reffering to Q-Free's sensor, it's battery size is 25 920\acrshort{mwh}, which means that with the calculated performance the sensor would not opperate for ten years.

\begin{table}[H]
\centering
\resizebox{\textwidth}{!}{%
\begin{tabular}{|l|l|l|l|l|}
\cline{1-2} \cline{4-5}
Transmits over 10 years & 87600 &  & Telenor (\acrshort{mwh}) & Telia (\acrshort{mwh}) \\ \cline{1-2} \cline{4-5}
\% of excellent transmits & 10 \% &  & 438 & 438 \\ \cline{1-2} \cline{4-5}
\% of transmits without faults & 80 \% &  &  &  \\ \cline{1-2} \cline{4-5}
\% of transmits with RAI & 67160 transmits &  & 13432 & 13432 \\ \cline{1-2} \cline{4-5}
\% of transmits without RAI & 2920 transmits &  & 1460 & 3504 \\ \cline{1-2} \cline{4-5}
 &  &  &  &  \\ \cline{1-2} \cline{4-5}
\% of bad transmits & 10 \% &  & 10512 & 10512 \\ \cline{1-2} \cline{4-5}
\% of reboots & 1 \% &  & 20.75 & 182.5 \\ \cline{1-2} \cline{4-5}
Sum &  &  & 25862.075 & 28068.5 \\ \cline{1-2} \cline{4-5}
\end{tabular}%
}
\caption{Long-term accumulated calculation}
\label{table:longtermcalculation}
\end{table}

\section{Combining the results}
I have presented many aspects of \acrshort{nb-iot} and tried to test them with Telenor and Telia, which produced results indicating the state of \acrshort{nb-iot}. This section gives a highlight of the pros and cons of \acrshort{nb-iot} in general and gives examples from the results. As stated in section \vref{section:challenges}, the long-term results are affected of early stage hardware and software, hence the main focus will still be at the short-term tests. The section is split into subsections, each focusing on a specific \acrshort{nb-iot} feature.

\subsection{Transmit process}
We have seen many transmit examples and in general they are all very similar to the specifications state. The transmit starts with a period of negotiation process, followed by the actual transmit and a optional \acrshort{rrc} period. With \acrshort{rai} set, Telenor and Telia performed very similar and according to the specifications. However, with \acrshort{rai} unset, Telia increases battery lifetime by using what they say is \acrshort{drx} within the \acrshort{rrc} period. This resolves in approximately 60\% reduced power usage and can affect your application if the transmits do not use \acrshort{rai}. For most application this will not be a problem since it is recommended to use \acrshort{rai} for transmits unless really necessary.

In general the results related to the transmit process are promising. There were some power spikes up to \textasciitilde250\acrshort{ma} in many transmits when having good or excellent signal. The spike was not present at every transmit, which leads me to believe that this is a network related issue. Moreover, the spike is very short and only happens once per transmit and will probably not be a major problem even though the chip uses more power than necessary.

\subsection{Coverage and latency}
Coverage and latency is closely related and the results show that \acrshort{nb-iot} meets the requirements of excellent coverage and latency below ten seconds. The specification states a coverage up to 35 kilometers, but this is without any obsticles, hence the coverage rate falls in city areas.

\subsection{Connection time}
Section \vref{ssection:reboottest}, displayed the results from the reboot tests. These tests showed what happened during the connection period and it was clear that there were big differences between Telenor and Telia. Telenor used a lot longer time to connect to the network, hence increasing the power usage. I did not manage to find the reason for the long connection time, but it is worth noting the difference, since this might affect the battery lifetime in certain areas where reconnection or reboots will occour often. This is neither a big concern, since the network should be stable and there is potentially no need to reboot the device.

\subsection{Uptime}
Through the long-term tests the stability of the networks was tested and at that time the results were poor. The uptime was not as good as expected and the device needed to be rebooted to reinitiate normal behavior. Telenor's network was more stable than Telia, but not by much.

\section{Future research}
\acrshort{nb-iot} is an interesting technology with great potenntial. For future research it would be interesting to explore certain aspects of the technology, for instance the overhead related to the transmit process. The article 'Overview of 3GPP Release 14 Enhanced NB-IoT', gives an overview of the new features and enhancements in \acrshort{3gpp} Release 14, which includes increased data rate, better support for mulit-carriers and lower power usage. In fact, 'in Release 14, a new UE power class with the maximum allowed output power reduced to 14 dBm was introduced to enable using smaller battery form factors for NB-IoT devices'\cite{Overview74:online}. Furthermore, it will be interesting to follow the development as the density increases.

Probably the most improvement at this time is \acrshort{ue} related. It will be very interesting to follow the development of Nordic Semiconductor's \acrshort{nb-iot} chip which should improve the flaws revealed by the tests in this thesis.

\section{Final remarks}
It has been a rewarding period working with \acrshort{nb-iot}. I believe that this is a promising technology which will improve many applications. \acrshort{iot} is growing and there are companies waiting for good \acrshort{lpwan}'s. As stated in section \vref{section:market}, there is a higher demand for \acrshort{lte-m1} in the U.S. which will postpone the production of \acrshort{nb-iot} hardware to around 2019. The good thing is that Telenor and Telia will have their \acrshort{nb-iot} implementations in production by 2018, meaning that the network is ahead of hardware production - which is unusual. There is one major drawback in the network today, which is the ability to investigate what happens when the device fails. The network is very much like a black box without any indication of the status, which can be frustrating for developers. The network providers must have documentation related to \acrshort{nb-iot}, thus the developers will have an easier time configuring their application to meet their requirements. Currently, there are too many unknown states and bugs in the network and the \acrshort{ue}, and this needs to be handled before the network providers production set their implementations. Most companies will have to wait for the next generation hardware which increases the stability, and at that point people can start creating solid applications which are battery efficient and easy to maintain without the high level of logic currently needed.

\backmatter{}
\printbibliography[]
\printglossary[type=\acronymtype]
\end{document}
